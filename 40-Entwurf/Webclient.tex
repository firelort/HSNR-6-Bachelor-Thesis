\section{Webclient} \label{sec:Webclient}
\subsection{SPA vs MPA} \label{subsec:SPA_vs_MPA}

\paragraph{Multi  Page Applications} \label{para:Multi_Page_Applications}
Multi Page Applications, kurz MPA, ist die klassische Architektur für Webanwendungen. Bei dieser Architektur wird für jeden Request (Anfrage) an den Webserver eine neue Seite inklusive Ressourcen wie Cascading Style Sheets (CSS)\footnote{Beinhalten Regeln für die Darstellung von unter anderem Webseiten}, JavaScript und Bildern geladen. Dieses kann mit einem Beispiel verdeutlicht werden.

Auf einer Shop-Seite befinden sich 10 Produkte inkl. Bild und Kurzbeschreibung. Wird ein Produkt ausgewählt, sendet der Client einen Request / eine Anfrage an den Webserver. Der Webserver antworte mit allen Ressourcen (siehe oben), welche für das Produkt benötigt werden. Der Client stellt dann aus den Ressourcen die Ansicht dar und das Produkt inklusive der Details ist für den Nutzer zu sehen.

Der Vorteil von MPAs ist die Optimierbarkeit für Suchmaschinen, das sogenannte SEO (Search Engine Optimization). Ein gutes SEO Rating sorgt dafür, dass die Webseite bei Suchmaschinen weit oben zu finden ist. Dies ist besonders für Webseiten und Shops wichtig, welche um Kunden konkurrieren. Anzuführen sind hier diverse Webshops und Zeitungen.

\paragraph{Single Page Applications} \label{para:Single_Page_Applications}
Die Single Page Applications, kurz SPA, stellt das genaue Gegenteil von MPA dar. Bei SPA besteht die Anwendung aus genau einem HTML-Dokument, dessen Inhalt bei Bedarf dynamisch nachgeladen wird. Dafür findet ein asynchroner Datenaustausch zwischen Client und Server statt, bei dem benötigte Ressourcen, wie Bilder, JavaScript und CSS ausgetauscht wird. Durch dieses Verfahren wird sicher gestellt, dass gleiche Elemente oder Ressourcen nicht erneut heruntergeladen werden müssen. Bei Änderungen werden nur Teile des DOMs\footnote{Das Document Object Model repräsentiert die Webseite als Baumstruktur} ersetzt und neu gerendert.

Die Interaktion mit dem DOM oder auch Virtual DOM kann selber entwickelt werden. Jedoch ist hierbei zu raten, auf bereits bestehende Frameworks wie Angular (Entwickelt unter der Leitung vom Angular Team bei Google), React (Entwickelt unter der Leitung von Facebook) oder Vue (Evan You und Core Team) zurück zugreifen. 

Der große Vorteil von SPA ist die Geschwindigkeit der Anwendung, da hier nur einzelne Teile ausgetauscht werden müssen. Auch bieten SPA den Vorteil, dass die Entwicklung von Front- und Backend entkoppelt wird. Das heißt, dass die Programmiere des Front- und Backends weitestgehend unabhängig von einander arbeiten können.

Die SEO Optimierung gestaltet sich schwieriger, da es sich um eine dynamische Anwendung handelt.
Zur Nutzung von SPA muss im Browser JavaScript verfügbar und aktiviert sein.

\paragraph{Zusammenfassung Vor- und Nachteile} \label{para:Zusammenfassung_Vor-_und_Nachteile}
\mbox{} % Damit die Tablle in der neuen Zeile angezeigt wird
\begin{table}[htb]
	\centering
	\begin{tabular}{rcc}
		& SPA & MPA \\
		Vorteile &
		\begin{minipage}[t]{0.4\textwidth}
			\begin{itemize}
				\item Sehr schnell, dank dynamischen nachladen
				\item Entkoppelung zwischen Front- und Backend
				\item Effizientes cachen von Daten
			\end{itemize}
		\end{minipage} &
		\begin{minipage}[t]{0.4\textwidth}
			\begin{itemize}
				\item MPA Architektur ist ausgereift
				\item MPAs sind Entwickler freundlich, da ein kleiner Technologiestack benötigt wird
				\item Ältere Browser werden unterstützt
				\item SEO ist einfacher zu implementieren
			\end{itemize}
		\end{minipage}\\
		& & \\ % Space zwischen Vor- und Nachteilen
		Nachteile &
		\begin{minipage}[t]{0.4\textwidth}
			\begin{itemize}
				\item JavaScript muss im Browser verfügbar sein
				\item Alte Browser werden nur teilweise unterstützt
				\item Herausfordernde SEO Implementierung
				\item Gefahr von XSS Attacken
			\end{itemize}
		\end{minipage} &
		\begin{minipage}[t]{0.4\textwidth}
			\begin{itemize}
				\item Anwendung sind weniger performant als MPAs
				\item Front- und Backend haben eine starke Kopplung
			\end{itemize}
		\end{minipage}
	\end{tabular}
\caption{Vor- und Nachteile SPA/MPA}
\label{tab:Vor-_und_Nachteile_SPA/MPA}
\end{table}

Für die Entwicklung der Anwendung entscheide ich mich für die Verwendung einer SPA. Dieses geschieht unter den Gesichtspunkten der Entkopplung zwischen Front- und Backend, der Performance der Anwendung und der Zukunftssicherheit, welche meiner Meinung nach für SPA besteht. Die Nachteile vom SPA betreffen meine Anwendung gering. So ist auf den Rechnern im Labor ein moderner Webbrowser installiert und in diesem JavaScript aktiviert. Auch handelt es sich um eine interne Anwendung, bei der die SEO Optimierung keine Rolle spielt. Einzig die Gefahr von XSS Attacken besteht, diese hoffe ich durch eine geeignete Wahl der Technologien sowie der Implementierung zu reduzieren.\cite{melnikSinglePageApplication2020}

\subsection{Mockups} \label{subsec:Mockup}
(todo: Mockups einfügen)