\section{Containerisierung} \label{sec:Containerisierung}
(todo: Weiter ausarbeiten, mit quellen versehen, andere Position)
Durch die Nutzung von Containerisierung ist es möglich die Anwendung agiler und skalierbarer zu betreiben. Auch werden so Abhängigkeiten zwischen den Komponenten reduziert und eine losere Kopplung erreicht, dies führt auch zu einer klareren Struktur und übersichtlicheren Programmierung.

Die Containerisierung wird mithilfe von Docker erreicht. Docker ist ein seit 2013 bestehende Open-Source Containerisierungssoftware, welche ein weite Verbreitung genießt. Sie war der de facto Standard für Containerisierung, wird in den letzten Jahren, jedoch durch Container-Orchestrierung Softwares, wie Kubernetes oder OpenShift, stückweise verdrängt.

Da eine Container-Orchestrierung viel mehr bietet als im Rahmen dieses Projektes benötigt wird und eine Verwaltung und Installation von solch einer Software nicht einfach ist, wird Docker, welches den Anforderung genügt, verwendet. So bietet Docker den Vorteil der einfacheren Bedienbarkeit und Installation sowie der benötigten Flexibilität.

Mithilfe von Docker Compose ist es möglich Applikationen, welche aus mehreren Containern bestehen auch Stack genannt, zu verwalten und zu betreiben.
Auch ist einfacher Abhängigkeiten zwischen den Containern zu modellieren und Verbindungen zwischen Containern herzustellen.

links: https://cloud.google.com/containers?hl=de, https://entwickler.de/leseproben/containerisierung-der-it-579775782.html