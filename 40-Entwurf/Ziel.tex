\section{Entwurfsziele} \label{sec:Entwurfsziele}

Bei dem Entwurf des neuen Systems sind neben den in der Analyse beschriebenen Anforderungen auch folgende Ziele beachtet worden:

\subsubsection{Beibehaltung der Features}
Die bereits implementierten Features Flagshop und Challenges sollen auch im neuen System verfügbar sein. Zusätzlich sollen die Studierenden angeregt werden, diese auch aktiv zu nutzen.

\subsubsection{Lose Kopplung}
Zwischen dem Scanner und dem Webserver soll eine lose Kopplung herrschen, damit die Entwicklung der beiden Komponenten unabhängig voneinander fortgesetzt werden kann.

\subsubsection{Datenhaltung in Datenbank}
Die Nutzung einer Datenbank sollte aufgrund zweier Überlegungen angestrebt werden: Erstens wären damit alle Daten an einem Ort gebündelt, zweitens könnte die Berechnung von Punkten an die Datenbank abgeben werden. Datenbanken sind unter anderem für solche Aufgaben geeignet.

\subsubsection{Modernisierung der GUI}
Das Graphical User Interface soll modernisiert werden, sodass es heutigen Standards entspricht. Auch sollen hierdurch die Verständlichkeit verbessert und die Challenges sowie der Flagshop besser platziert werden.

\subsubsection{Einheitliche Programmiersprache}
Eine einheitliche Programmiersprache sollte, sofern dies möglich ist, genutzt werden. Das Betreiben der Komponenten wird erleichtert, da nicht zwei verschiedene Programmiersprachen und/oder Umgebungen installiert werden müssen. Auch erleichtert es die Programmierung, wenn nur eine entwickelnde Person parallel an den Komponenten arbeitet. Die Gefahr von falschen Syntaxen und verschiedenen Konventionen kann dadurch reduziert werden. Sollte die Anwendung durch mehrere entwickelnde Personen implementiert, gewartet sowie auf verschiedenen Systemen betrieben werden, ist dieses Ziel nichtig.

\subsubsection{Module sparsam nutzen}
Bei der Implementierung der Software sollte, soweit dies notwendig und sinnvoll ist, auf bereits vorhandene Module und Frameworks zurückgegriffen werden. Durch die Minimierung von Abhängigkeiten ist die Wartung und Verwendung der Software durch Dritte leichter möglich. Auch wird die Gefahr von Fehler auslösenden Updates minimiert. Bei der Nutzung von Modulen und Frameworks sollte auf deren Verbreitung und Wartung geachtet werden, um eine Nutzung von potenziell inaktiven Modulen/Frameworks mit eventuell vorhanden Schwächen oder Sicherheitslücken zu reduzieren.

\subsubsection{Containerisierung}
Die Anwendung soll mit möglichst kleinem Wartungsaufwand überall benutzbar sein. Um dies zu gewährleisten, sollte eine Containerisierung genutzt werden. Bei der Nutzung ist darauf zu achten, ob und mit welchen Einschränkungen diese einsetzbar ist.

\subsubsection{Ressourcen schonen}
Um die Ressourcen des Servers zu schonen, sollten die nicht benötigten Komponenten abgeschaltet werden. Hierbei ist der Scanner hervorzuheben, welcher nur während des Praktikums laufen muss. 