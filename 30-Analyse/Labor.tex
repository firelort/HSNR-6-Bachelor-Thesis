\section{Ausstattung Labor}
\label{sec:Ausstattung_Labor}

Das Praktikum wird im Labor für Echtzeitsysteme (EZS Labor) der Hochschule Niederrhein durchgeführt.

Das Labor ist mit acht Gruppenarbeitsplätzen für Studenten sowie Arbeitsplätzen für die Betreuer und Mitarbeiter ausgestattet. Ein Arbeitsplatz der Betreuer kann zu einem neunten Gruppenarbeitsplatz umfunktioniert werden.

An einem Gruppenarbeitsplatz können 2 Studierende gleichzeitig arbeiten, da diese mit einem leistungsfähigem Desktop-PC und einem Raspberry Pi\footnote{Einplatinencomputer mit der Größe einer Kreditkarte} sowie den dazugehörigen Peripheriegeräten (Maus, Tastatur \& Monitor) ausgestattet sind.
Auf den Desktop-PCs ist Ubuntu\footnote{Ubuntu ist eine freie Linux Distribution auf Basis von Debian} und auf den Raspberry Pis ist Raspbian\footnote{Abwandlung von Debian für den Raspberry Pi} als Betriebssystem installiert.

Auf den Desktop-PCs ist die Software VirtualBox der Firma Oracle installiert. Diese Software ermöglicht es auf dem Rechner einen weiteren Rechner zu virtualisieren. Dieser weitere PC wird Guest genannt und kann den Host, den Rechner auf dem die Software VirtualBox läuft, nicht schädigen oder beeinflussen. Sollte auf dem Guest ein Virus aktiv werden, kann dieser nicht den Host angreifen. Hierbei sollte beachtet werden, dass die Software VirtualBox Fehler haben kann oder der Nutzer Einstellungen getroffen hat, sodass der Host doch angreifbar ist.

Neben diesen Rechner steht ein Linux Server zur Verfügung, auf welchem das Auswertungs- und Überwachungssystem läuft.

Alle Rechner, auch die Guest System der Studentengruppe, sind untereinander via Ethernet verbunden.

Auch steht ein Beamer zur Verfügung auf dem die aktuelle Spielübersicht dargestellt werden kann.