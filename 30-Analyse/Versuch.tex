\section{Versuch \textquote{Catch me, if you can}}
\label{sec:Versuch}

Der zweite der drei Versuche \textquote{Catch me, if you can} wird im Rahmen eines Wettbewerbs zwischen den teilnehmenden Studierendenteams ausgetragen. Der Wettbewerb ist an ein Capture the Flag (CTF) angelehnt. Bei einem klassischen CTF erhält der Spieler durch das Lösen von Aufgaben einen bestimmten Text. Dieser wird Flag genannt. Die Aufgaben können das Lösen einer Art Schnitzeljagd, eine einfache Programmierung, aber auch das Hacken mehrerer entfernter Rechner umfassen. Anders als beim klassischen CTF werden bei \textquote{Catch me, if you can} die Flags auf allen teilnehmenden Systemen verteilt. \cite{tanWhatCTFHow2020} Die Studierenden können diese durch das Analysieren ihres eigenen Gastsystems sowie durch den Angriff auf fremde Gastsysteme erhalten. Besonderheit hierbei sind die Strafpunkte für den Verlust einer Flag an gegnerische Studierendenteams. Wie beim klassischen CTF können die Studierenden Flags und Punkte durch zentrale Aufgaben erhalten.

Der Versuch ist in drei Phasen untergliedert.
\begin{enumerate}
	\item Vorbereitung
	\item Wettbewerb
	\item Abschluss
\end{enumerate}

\subsubsection{Vorbereitung}
Die Studierenden erhalten circa 30 Minuten Zeit, um ihr Gastsystem in Betrieb zu nehmen und sich mit diesem vertraut zu machen. Hierbei sollten die Schwachstellen in den vorhandenen Diensten abgesichert und der Zugriff durch andere Studierende verhindert werden. Während dieser Zeit dürfen die Studierenden andere Systeme nicht angreifen. Auch ist es möglich, in dieser Zeit Flags auf dem eigenen System der Studierenden zu suchen. Da der Ablageort der Flags auf allen Systemen gleich ist, kann diese Information im Spielverlauf bei einem eigens initiierten Angriff schneller Flags einbringen.

\subsubsection{Wettbewerb}
Die Wettbewerbsphase selbst dauert circa 140 Minuten. In dieser Zeit sind Angriffe auf fremde Gastsysteme erlaubt und ausdrücklich erwünscht. Eine weitere Absicherung ist ebenfalls möglich. Das System sollte auf fremde Aktivitäten hin überwacht werden. Diese Aktivitäten sollten schnellstmöglich unterbunden werden, da die Angreifer Flags entwenden können und so dem Team Strafpunkte einbringen. Auch kann die Zeit für die Lösung von zur Verfügung stehenden Challenges sowie die Nutzung des Flagshops genutzt werden. Der Flagshop sowie die Challenges werden im nächsten Kapitel aufgegriffen.

\subsubsection{Abschluss}
Nach Ende der Wettbewerbsphase müssen die Studierenden ihre Angriffe einstellen und eine weitere Flagabgabe ist nicht mehr möglich. Die Studierenden erstellen für ihren anzufertigen Versuchsbericht einen Screenshot der Punkteübersicht. Eine Nachbesprechung ist optional und auf maximal 30 Minuten begrenzt.

Während des Wettbewerbs gelten die aufgelisteten Regeln. Es handelt sich hierbei nur um einen Auszug der für die Bachelorarbeit relevanten Regeln.
\begin{itemize}
\item Der Gameserver darf nicht angegriffen werden
\item Es dürfen nur die Gastsysteme angegriffen werden
\item Das Passwort des Logins \textit{gamemaster} darf nicht zurückgesetzt werden
\item Der SSH-Server muss für alle erreichbar sein
\item Flags dürfen nicht modifiziert oder gelöscht werden
\item Sämtliche Dienste müssen für den Gameserver erreichbar bleiben
\item ICMP-Pakete (ping) dürfen nicht blockiert werden.
\end{itemize} \cite[S.9]{quadePraktikumITSecurity2017} \cite[S.10-11]{sosnaKonzeptionUndRealisierung2010}