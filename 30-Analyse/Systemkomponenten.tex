\section{Systemkomponenten}\label{sec:Systemkomponenten}

\subsection{Komponenten des Servers}\label{subsec:Komponente_des_Servers}
Im Folgenden werden die verschiedenen Komponenten des Auswertungs- und Überwachungssystems in der derzeitigen Implementierung untersucht. Dabei werden Rückschlüsse auf Anforderungen gezogen sowie Schwachstellen und Verbesserungsmöglichkeiten herausgearbeitet.

\subsubsection{Scanner}\label{subsubsec:Scanner}
Der Scanner prüft in regelmäßigen Abständen die auf den Gastsystemen der Studierenden installierten Dienste und speichert das Ergebnis ab. Die Abstände können beim Starten des Spieles eingestellt werden. Die folgenden Dienste werden pro Team geprüft.

\paragraph{ScanUp}\label{para:ScanUp}
Die Aufgabe dieses Scans besteht darin zu prüfen, ob das Gastsystem noch für den Server erreichbar ist. Sollte das Gastsystem nicht erreichbar sein, wird hierfür ein Strafpunkt vergeben. Aus technischer Sicht wird das Linux Kommando \textit{ping} verwendet. Anhand des Rückgabewertes kann nachvollzogen werden, ob der Server das Gastsystem erreichen konnte.

\paragraph{ScanBubble}\label{para:ScanBubble}
Auf dem Gastsystem läuft ein selbst programmierter Bubble Server, welcher Flags unter Nutzung des Telnet Protokolls bereitstellt. Nachdem eine Flag abgeholt worden ist, stellt der Dienst für eine bestimmte Zeit (Timeout) keine weitere Flag bereit. Der Bubble Server nimmt Anfragen über den Port \textit{12321} für unverschlüsselte Flags und Port \textit{12322} für verschlüsselte Flags entgegen. Die Scan-Operation überprüft, ob eine Telnet Verbindung zu dem Port \textit{12321} möglich ist, in dem die Operation eine Telnet Verbindung öffnet und prüft, ob die Verbindung erfolgreich war.

\paragraph{ScanWebUp}\label{para:ScanWebUp}
Jedes Gastsystem stellt unter Zuhilfenahme eines Apache Web Servers und php-Dateien Webseiten und Daten bereit, welche über einen Web Client abgerufen werden können. Dazu muss auf Port \textit{80} der HTTP- und auf Port \textit{443} der HTTPS-Dienst laufen und erreichbar sein. Dieses verifiziert die Scan-Operation indem sie eine Socket Verbindung zu den Ports \textit{80} und \textit{443} öffnet und das Ergebnis prüft.

\paragraph{ScanSQLInjectUp}\label{para:ScanSQLInjectUp}
Dieser Scan prüft, ob die Login-Seite des Teams, auf der die SQL-Injection Schwachstelle implementiert ist, erreichbar und benutzbar ist. Die Operation sendet hierzu eine valide Kombination aus Nutzername und Passwort an den Webserver. Das Ergebnis wird dann mit dem erwarteten Resultat verglichen.

\paragraph{ScanSQLInjectSave}\label{para:ScanSQLInjectSave}
Ähnlich wie bei der Operation ScanSQLInjectUp (\ref{para:ScanSQLInjectUp}) wird geprüft, ob Ergebnis und Erwartung übereinstimmen. Besonderheit hierbei ist, dass statt einer validen Kombination aus Nutzernamen und Passwort eine sogenannte SQL-Injection (wird im \autoref{subsec:Komponente_des_Clients} genauer erläutert) im Nutzernamen übergeben wird.  
Sollte die Anfrage alle gespeicherten Nutzerdaten zurückgeben, ohne dass eine Authentifizierung stattfindet, ist die SQL-Injection weiterhin möglich.

\paragraph{ScanXSSSave}\label{para:ScanXSSSave}
Diese Scan-Operation prüft, ob die auf dem Gastsystem implementierte XSS Schwachstelle behoben wurde. Dazu wird die Webseite mit präpariertem Inhalt aufgerufen. Auf die Vorgehensweise wird im \autoref{subsec:Komponente_des_Clients} eingegangen.
In der Rückgabe wird geprüft, ob dieser ungefiltert auf der Webseite zu finden ist. Sollte dies der Fall sein, ist die XSS Schwachstelle nicht oder unzureichend von den Studierenden abgesichert worden.

\paragraph{ScanSQLSave}\label{para:ScanSQLSave}
Bei diesem Scan wird kontrolliert, ob der Login mit dem auf allen Systemen voreingestellten Passwort \textit{toor} für den SQL-Account \textit{root} möglich ist. Sollte dieser möglich sein, haben die Studierenden dieses unsichere Passwort nicht geändert. Des Weiteren wird geprüft, ob das htaccess Passwort, welches die phpMyAdmin Anwendung schützen soll, geändert worden ist.

\paragraph{ScanFTPSave}\label{para:ScanFTPSave}
Auf dem Client System läuft ein FTP Server, welcher ohne Login (Nutzername \& Passwort) Daten bereitstellt. Der Scan prüft, ob ein sogenannter Anonymous Login möglich ist, indem eine FTP Verbindung ohne Login aufgebaut wird. Sollte die Verbindung erfolgreich sein, ist der Anonymous Login immer noch möglich.

\paragraph{ScanTelnetSave}\label{para:ScanTelnetSave}
Ein Telnet Server horcht auf Verbindungen auf Port \textit{23}. Da dieser Dienst nicht benötigt wird, soll er durch die Studierenden abgeschaltet oder deinstalliert werden. Die Scan-Operation prüft, ob eine Verbindung mithilfe des Telnet Protokolls auf Port \textit{23} möglich ist. Dazu wird eine Verbindung zu Port \textit{23} aufgebaut und das Resultat geprüft.

\subsubsection{Generierung von Flags}\label{subsubsec:Generierung_von_Flags}

Derzeitig erfolgt die Generierung der Flags sowohl auf den Gastsystemen als auch auf dem Auswertungs- und Überwachungssystem. Dies ist notwendig, da ansonsten eine Überprüfung der Gültigkeit der Flags und Verrechnung der Punkte nicht durchgeführt werden kann. Die Flags werden durch einen Algorithmus generiert. Dieser erzeugt pro Team eine bestimmte Anzahl an Flags. 

Dazu wird die Flag mithilfe der Streuwertfunktion (Hashfunktion) MD5 und der Eingabe, einem sogenannten seed, berechnet. Eine Hashfunktion bildet aus einer Eingabe variabler Länge eine Ausgabe mit einer festen Länge. Bei identischer Eingabe wird immer der gleiche Ausgabewert berechnet. Des Weiteren ist es bei einer guten Hashfunktion nicht möglich, von der Ausgabe auf den Eingabewert zu schließen. \cite{menezesHandbookAppliedCryptography1996}

Der in der Anwendung genutzte seed setzt sich aus der Verkettung von \textit{Salt}, \textit{IP-Adresse}, dem Wort \textquote{\textit{Aufgabe}} und einem \textit{Zähler} zusammen.

Ein Salt wird benötigt, um den Flags eine Lebenszeit zu geben. In der derzeitigen Implementierung enthält der Salt das aktuelle Jahr sowie das jeweilige Semester. So sind nur Flags des aktuellen Semesters gültig und werden vom Auswertungs- und Überwachungssystem akzeptiert. Einer Verwendung von Flags aus vorherigen Semestern wird somit effektiv vorgebeugt.

Die IP-Adresse stellt hierbei den Bezug zum jeweiligen Team dar.

Der String \textquote{Aufgabe} wird als Geheimnis verwendet, um das Fälschen von Flags zu erschweren und bestenfalls zu verhindern.

Damit pro Team mehrere eindeutige Flags generiert werden können, wird ein sogenannter Zähler genutzt. Dieser Zähler ist auf 0 initialisiert und wird pro generierter Flag um eins erhöht, bis die benötigte Anzahl an Flags generiert ist. \cite[S.48]{sosnaKonzeptionUndRealisierung2010}

\subsubsection{Webserver}\label{subsubsec:Webserver}

Der Webserver stellt die GUI (Graphical User Interface) für die Studierenden und betreuenden Personen dar. Hier kann der aktuelle Punktestand angesehen werden. Auch wird in der GUI dargestellt, welches Team welchen Service abgesichert hat, inklusive der Negativpunkte für nicht abgesicherte Dienste, und wie viele Strafpunkte das jeweilige Team erhalten hat.

Neben diesen Darstellungen befinden sich auf dem Server ein sogenannter Flagshop und diverse Challenges, mit denen Studierende weitere Flags erhalten können.

Die betreuenden Personen haben die Möglichkeit, über die Web-GUI ein neues Spiel anzulegen und das Spiel zu starten oder zu stoppen. Auch kann von dem Spiel ein Backup erstellt werden. Neben diesen Funktionen zur Spielsteuerung können an die Teams Strafen für unfaires oder regelverletzendes Verhalten verteilt werden. Diese nehmen direkten Einfluss auf die Punkte des jeweiligen Teams. Außerdem besteht die Möglichkeit, weitere Benutzer für das Administrationsinterface zu registrieren.

\paragraph{Flagshop} \label{para:Flagshop}
Der Flagshop ermöglicht es den Studierenden, weitere Flags mit ihren Punkten zu kaufen. Der Kauf von Flags lohnt sich schon, da die verkauften Flags mehr Punkte bringen als sie kosten. Um einen Einkauf im Flagshop durchzuführen, müssen die Teams vorher einen Account erstellen. Die Registrierung erfragt neben dem benötigten Benutzernamen und Passwort auch für den Flagshop irrelevante Daten ab. Diese ähneln persönlichen Informationen, welche bei den meisten Onlineshops angegeben werden müssen. Das Format, hier die Repräsentation als Zahl oder String sowie die Länge, und die Erforderlichkeit der Daten werden nur im HTML-Formular festgelegt. Durch eine Manipulierung des Formulars kann dieses mit nicht konformen oder nicht vorhanden Daten abgesendet werden. Für jede der nicht vorhandenen oder nicht konformen Informationen erhält der Studierende eine Flag. Daneben wird die Güte des angegebenen Passwortes anhand von Länge und Anzahl an Sonderzeichen, Groß- und Kleinbuchstaben sowie Ziffern bewertet und mit Flags belohnt.

Nach der Registrierung können die Studierenden sich für ihre Punkte Flags kaufen. Dazu stehen zwei Pakete mit 8 bzw. 6 Flags für den Preis von jeweils 4 Punkten pro Paket zur Verfügung. Dieser Preis kann auf zwei Arten reduziert werden. 
Einmal müssen sich die beiden Pakete gleichzeitig im Warenkorb befinden und die Identifikationsnummern (ID) dieser müssen auf nicht vorhandene Nummern gesetzt werden. Die Manipulation resultiert in einem reduzierten Preis von 4 Punkten für beide Pakete. Dies ist extra im Flagshop einprogrammiert und soll die Studierenden auf Manipulation von IDs aufmerksam machen.
Durch die zweite Art ist es möglich, die Pakete kostenlos zu erhalten. Dazu muss im Warenkorb das sogenannte \textit{hidden input} Feld, in dem der aktuelle Preis des Warenkorbs gespeichert wird, auf 0 gesetzt werden. Dann berechnet der Flagshop für den Kauf einen Preis von 0 Punkten. \cite[S. 63]{abtsUeberarbeitungUndErweiterung2016}

Ein \textit{hidden input} Feld wird in der Repräsentation eines HTML-Dokumentes nicht angezeigt, kann jedoch durch die Entwicklertools der Browser betrachtet und verändert werden. \cite{w3schoolsHTMLHiddenInput}

Auf diese Weise ist es auch möglich, einen negativen Preis festzulegen und dem eigenen Team Punkte zuzuschreiben, da die derzeitige Implementierung nicht prüft, ob der von dem Nutzer eingegebene Preis kleiner als 0 ist, sondern ob dieser gleich 0 ist. Bei richtiger Implementierung würde ein Preis kleiner 0 geprüft und korrigiert.

\paragraph{Challenges} \label{para:Challenges}
Derzeitig sind fünf Challenges implementiert, welche vom System in zufälliger Reihenfolge an interessierte Teams verteilt werden. Eine abgeschlossene oder abgebrochene Challenge, durch das Neuladen der Webseite oder durch das Betätigen der Zurück-Taste, kann nicht wiederholt werden. Eine Challenge kostet 10 Punkte. Nach erfolgreichem Abschließen einer Challenge gibt es 10 Punkte plus eine gewisse Anzahl an Punkten für das Absolvieren der Aufgabe. Die folgenden Challenges sind implementiert. \cite[S.19-20]{abtsUeberarbeitungUndErweiterung2016}

\subparagraph{Aufgabe 1: robots.txt}\label{subpara:Aufgabe_1_robots.txt}
Die Studierenden sollen in dieser Challenge lernen, dass die \textit{robots.txt} Datei keinen Zugriffsschutz darstellt. Diese bittet nur Suchmaschinen, die angegeben Verzeichnisse und Dateien nicht zu indexieren. Aus der \textit{robots.txt} Datei können Informationen zu versteckten Dateien und Verzeichnissen erhalten werden. Die Studierenden sollen über die \textit{robots.txt} Datei einen vorhandenen Ordner finden. In diesem befindet sich die Lösung zur Challenge.

\subparagraph{Aufgabe 2: JavaScript-Login-Bypass}\label{subpara:Aufgabe_2_JavaScript-Login-Bypass}
Bei dieser Challenge ist das benötigte Passwort zur Lösung der Challenge als Klartext im Quelltext versteckt. Das Verstecken wird derzeitig mit einer Meldung wie \textquote[\cite{abtsUeberarbeitungUndErweiterung2016}]{Seitenquelltext deaktiviert} und vielen Leerzeilen realisiert. Im Inspector von Firefox Version 78.0.1 und Chromium Version 83.0.4103.116 ist dies nicht möglich, da die Leerzeilen entfernt werden und das Passwort daher oben im Quelltext zu sehen ist.

\subparagraph{Aufgabe 3: Form-Modification}\label{subpara:Aufgabe_3_Form-Modification}
In dieser Challenge sollen die Studierenden verstehen, dass auch die Werte von Drop-Down-Menüs, Checkboxen und Radio-Buttons durch Manipulation auf nicht vorgegebene Werte geändert werden können. Deshalb müssen Nutzereingaben stets auch serverseitig überprüft werden, da hier die Regeln der Überprüfung nicht durch die Nutzer verändert werden können.

Die Aufgabe besteht darin, einen bestimmten Login-Namen aus einem Drop-Down-Menü auszuwählen. Da der geforderte Name nicht in dieser Liste ist, müssen die Studierenden das HTML Formular so manipulieren, dass er auswählbar ist.

\subparagraph{Aufgabe 4: JavaScript-Substrings}\label{subpara:Aufgabe_4_JavaScript-Substrings}
Das Passwort, welches die Studierenden eingeben müssen, wird clientseitig mithilfe einer JavaScript Funktion geprüft. Damit das Passwort nicht im Klartext im Quelltext steht, wird dieses verschleiert. So werden drei Strings Zeichen für Zeichen verglichen. Sollte ein Zeichen in mindestens zwei der drei Strings übereinstimmen, dann gehört dieses Zeichen zum Passwort. Im Anschluss wird geprüft, ob das generierte Passwort gleich dem ist, das die Studierenden als Passwort gegeben haben.

\subparagraph{Aufgabe 5: URL-Hex-Injection}\label{subpara:Aufgabe_5_URL-Hex-Injection}
Die Studierenden sollen an geheime Informationen in einem Ordner gelangen, welcher nach einem Wert aus dem Hexadezimalsystem (Basis 16 statt Basis 10, wie beim Dezimalsystem) benannt ist. Die Aufgabe soll zeigen, dass Ordner, die nach einem Hexadezimalwert benannten sind, auf diese Art und Weise nicht vor Zugriffen geschützt werden können, da das Präfix Zeichen \textit{\%} für Hexadezimalzahlen selber durch einen Hexadezimalwert dargestellt werden kann.

\subsubsection{Abgabe von Flags}\label{subsubsec:Abgabe_von_Flags}
Um Flags abgeben zu können, müssen die Studierenden sich mit ihren Zugangsdaten, welche sie durch das Lösen des Hackits erhalten haben, in der Web-GUI anmelden. Dort ist es möglich, in einem Input-Feld eine Flag synchron abzugeben. Das bedeutet, dass nach jeder Abgabe die Webseite neu geladen wird. Des Weiteren ist es nicht möglich, mehrere Flags gleichzeitig abzugeben.

\subsection{Komponenten des Clients}\label{subsec:Komponente_des_Clients}
Da sich die Bachelorarbeit mit der Modernisierung des Auswertungs- und Überwachungssystem beschäftigt, sind nur die für die Bachelorarbeit wichtigen Komponenten des Clients beschrieben.

\subsubsection{Webserver des Clients}\label{subsubsec:Webserver_des_Clients}
Auf den Clients wird ein Webserver mit einigen extra implementierten Schwachstellen betrieben. Diese sollen während des Versuches durch die Studierenden behoben werden.

Der Webserver stellt eine Art Kundenbewertungsformular mit implementierter XSS Schwachstelle bereit. In der verwendeten Implementierung wird die Eingabe des Nutzers in das HTML-Dokument geschrieben. Durch die Schwachstelle wird eine Nutzereingabe ungefiltert in das HTML Formular übernommen und potenziell schädlicher Code wird vom Browser verarbeitet. Dieser Code kann dann beispielsweise vertrauliche Informationen, wie Cookies, Session Tokens oder persönliche Daten auslesen und den Angreifern übermitteln. \cite{ruettenSicherheitWebanwendungen2007}

Eine weitere Schwachstelle stellt der sogenannte \textquote{Login zum Membersbereich} mit der implementierten SQL-Injection Schwachstelle dar. 
Bei einer SQL-Injection versucht ein Angreifer den verwendeten SQL-Befehl so zu erweitern, dass dieser neben der eigentlichen Abfrage an die Datenbank auch einen vom Angreifer vorbereiteten SQL-Befehl ausführt. Die Erweiterung des SQL-Befehls erfolgt, indem die Eingabe, welche in den SQL-Befehl aufgenommen wird, das Zeichen für das Ende des SQL-Befehls inklusive des vom Angreifer intendierten SQL-Befehls enthält. Über diesen Angriff können dann unerlaubterweise Daten ausgelesen werden, aber auch eine Löschung der Daten ist denkbar. \cite{bachfeldGiftspritze2004}

In der verwendeten Implementierung werden der Benutzername und das Passwort ungefiltert in den SQL-Befehl übernommen. Diese Schwachstelle lässt sich erst beheben, wenn die Gruppe die SQL-Injection bei sich selber erfolgreich durchgeführt hat.

Neben diesen Schwachstellen gibt es eine Registrierung für den Flagshop. Diese erfordert einige Eingaben wie Name, Alter, Postleitzahl und vieles mehr. Die Eingaben sind im HTML-Formular als Pflicht markiert und haben eine Vorgabe in der Form. Ein Absenden ist ohne Angabe dieser Daten nicht möglich. Die Studierenden erhalten jedoch für jede nicht getätigte und für jede nicht der Form entsprechenden Angabe Flags nach der Registrierung. Dies ist möglich, da das HTML-Formular durch die Studierenden geändert werden kann und der Server nur die Angaben bezüglich Passwort und Nutzername prüft. Diese beiden Angaben werden genutzt, um sich am Flagshop des Servers anzumelden und Flags zu erwerben. \textit{(Siehe: \autoref{para:Flagshop} Flagshop)}

Außerdem stellt der Webserver eine Bildergalerie zur Verfügung. In dieser befinden sich zum Spielstart zwei normal erscheinende Bilder. Entgegen dem Anschein sind in diesen Flags verschleiert. Die Steganografie wird durch die Kombination eines Bildes und eines zip-Archivs erreicht. \cite{abtsUeberarbeitungUndErweiterung2016}