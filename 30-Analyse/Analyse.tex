\label{chap_text:Analyse}
In diesem Kapitel werden die Voraussetzungen im Labor vorgestellt, die derzeitige Implementierung des Auswertungs- und Überwachungssystems beleuchtet und kurz auf einen überwachten Client sowie dessen Schnittstellen zum System eingegangen.

Die Lehrveranstaltung \textquote{IT-Sicherhiet} der Hochschule Niederrhein beschäftigt sich mit den Gefährdungszielen Integrität von Daten, Nutzbarkeit von Systemen und der (digitalen) Privatsphäre vertraut machen.\cite{quadeWasSieSchon2019}

Die Lehrveranstaltung ist in Vorlesung, Übung und Praktikum untergliedert. Durch die Praktika sollen sich die Studierende praktisch mit dem Thema beschäftigen und beweisen, dass sie die in der Lehrveranstaltung vermittelten Themen verstanden haben.

Das Praktikum \textquote{Catch  me, if you can} ist das zweite von drei Praktika, welches die Studierenden als Voraussetzung für die Klausurteilnahme erfolgreich absolvieren müssen.

Bevor die Studierenden am Praktikum teilnehmen können, müssen diese ein sogenanntes Hackit\footnote{Aufgabe aus dem Bereich IT-Security / Hacking} lösen und das erhalte Passwort zum nächsten Termin mitbringen. Ohne dieses Passwort ist die Teilnahme am Praktikum nicht möglich.

