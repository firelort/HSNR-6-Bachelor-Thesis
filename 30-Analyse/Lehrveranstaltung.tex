\section{Lehrveranstaltung IT-Sicherheit} \label{sec:Lehrveranstaltung_IT-Sicherheit}

Das Pflichtmodul IT-Sicherheit (ITS) ist in drei Veranstaltungen gegliedert.\cite[S.30]{hochschuleniederrheinModulhandbuchVollzeitBA2019}
\begin{itemize}
	\item Vorlesung (2 Semesterwochenstunden)
	\item Übung (1 Semesterwochenstunde)
	\item Praktikum (1 Semesterwochenstunde)
\end{itemize}

\paragraph{Vorlesung}\label{para:Vorlesung}
Die Vorlesung wird im wöchentlichen Turnus angeboten und behandelt grundlegendes Wissen zu IT-Sicherheit unter anderem in den Bereichen Gefährdung, Gegenmaßnahmen aber auch im Bereich rechtliche Gegebenheiten. Es werden Beispiele gegeben, bei welchen die angesprochen Themen gar nicht oder in einem ungenügenden Zustand umgesetzt worden sind. Die Vorlesung wird von den Veranstaltungen \textit{Übung} (freiwillig) und \textit{Praktikum} (verbindlich) ergänzt.

\paragraph{Übung}\label{para:Übung}
Die Übungen sind freiwillig und werden im zweiwöchentlichen Turnus á 2 Stunden angeboten. Diese ermöglichen den Studierenden den durch die Vorlesung und das Selbststudium vermittelt Stoff zu vertiefen und zu festigen. Auch können dort praktische Erfahrungen gesammelt werden, von denen die Studierenden unter anderem im Praktikum profitieren können.

\paragraph{Praktikum}\label{para:Praktikum}
Die Praktika finden im monatlichen Turnus (3x im Semester) á 4 Stunden statt. Das Bestehen aller drei der Praktika erhalten die Studierenden ihre Klausurzulassung. Das Praktikum muss vorbereitet werden, dazu erhalten die Studierenden vor dem Praktikum ein Hackit. Nur mit erfolgreichem Absolvieren des Hackits ist es möglich am nächsten Praktikum teilzunehmen.\cite{quadePraktikumITSecurity2017}