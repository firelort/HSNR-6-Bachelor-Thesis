\section{Lehrveranstaltung IT-Sicherheit} \label{sec:Lehrveranstaltung_IT-Sicherheit}

Das Pflichtmodul IT-Sicherheit (ITS) ist in drei Veranstaltungen gegliedert. \cite[S.30]{hochschuleniederrheinModulhandbuchVollzeitBA2019}
\begin{itemize}
	\item Vorlesung (2 Semesterwochenstunden)
	\item Übung (1 Semesterwochenstunde)
	\item Praktikum (1 Semesterwochenstunde)
\end{itemize}

\subsubsection{Vorlesung}
Die Vorlesung wird im wöchentlichen Turnus angeboten und behandelt grundlegendes Wissen zur IT-Sicherheit unter anderem in den Bereichen Gefährdung, Gegenmaßnahmen, aber auch im Bereich rechtliche Gegebenheiten. Es werden Beispiele aufgezeigt, bei denen die angesprochenen Themen gar nicht oder in einem ungenügenden Zustand umgesetzt worden sind. Die Vorlesung wird von den Veranstaltungen \textit{Übung} (freiwillig) und \textit{Praktikum} (verbindlich) ergänzt.

\subsubsection{Übung}
Die Übungen sind freiwillig und werden im zweiwöchentlichen Turnus á 2 Stunden angeboten. Diese ermöglichen es den Studierenden, den durch die Vorlesung und das Selbststudium vermittelten Stoff zu vertiefen und zu festigen. Auch können dort praktische Erfahrungen gesammelt werden, von denen die Studierenden unter anderem im Praktikum profitieren können.

\subsubsection{Praktikum}
Das Praktikum findet im monatlichen Turnus (3x im Semester á 4 Stunden) statt. Es ist in drei Laborversuche untergliedert. Nach Bestehen aller Versuche erhalten die Studierenden ihre Klausurzulassung. Jeder Versuch des Praktikums muss vorbereitet werden, dazu erhalten die Studierenden vor dem Versuch ein Hackit\footnote{Knobelaufgabe aus dem Bereich IT-Sicherheit / Hacking}. Nur mit erfolgreichem Absolvieren des Hackits ist es möglich, am nächsten Versuch teilzunehmen. \cite{quadePraktikumITSecurity2017}