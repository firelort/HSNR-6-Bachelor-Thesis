\documentclass[a4paper, 12pt, parskip=half]{scrbook}

% Einfaches Copy and Paste ermöglichen
\usepackage{cmap}

% deutsche Silbentrennung
\usepackage[ngerman]{babel}

% deutsche Umlaute
\usepackage[utf8]{inputenc}
\usepackage[T1]{fontenc}

\usepackage{csquotes}

% Keine extra Leerzeichen nach einem Punkt
\frenchspacing

% Schriftart
\usepackage{times}

% Stichwortverzeichnis
\usepackage{imakeidx}
\makeindex

% Paket für Änderungen an Kopf- und Fußzeile
\usepackage{scrlayer-scrpage}
% Bisherige Einstellungen für Kopf- und Fußzeilen löschen:
\clearpairofpagestyles
% Einstellungen für die Fußzeile, die Kopfzeile wird nicht verwendet
	% Einstellungen nur für die rechten Seiten
	% Layout: |  Seite
\rofoot{\textbf{$\mid$~~\pagemark}}
	% Einstellungen nur für die linken Seiten
	% Layout: Seite  |
\lefoot{\textbf{\pagemark~~$\mid$}}

% Paket für code Beispiele
% \usepackage{listings}

% Paket für Komma getrennte Fußnoten
\usepackage[multiple]{footmisc}

% Paket für Verlinkungen
\PassOptionsToPackage{hyphens}{url}\usepackage[breaklinks]{hyperref}
\makeatletter
\g@addto@macro{\UrlBreaks}{\UrlOrds}
\makeatother

\renewcommand{\UrlBreaks}{\do\/\do\-\do\_\do\&\do\?}	% allows URL breaking on /, -, _ and &

% Eigene Commands
\newcommand{\thesisTitle}{Vorläufiges Thema}
\newcommand{\thesisSubject}{Vorläufiges Subject}
\newcommand{\thesisAuthor}{Robert Hartings}
\newcommand{\Matrikelnummer}{1164453}


% Literaturverzeichnis einrichten
\usepackage[style=alphabetic, citestyle=alphabetic]{biblatex}
\addbibresource{thesis.bib}

\hypersetup{
	unicode = true, % allows to use characters of non-Latin based languages in Acrobat’s bookmarks 
	pdftitle = {\thesisTitle}, % define the title that gets displayed in the "Document Info" window of Acrobat 
	pdfauthor = {\thesisAuthor},
	pdfsubject = {\thesisSubject},
	pdfkeywords = {ITS2, hack-me-if-you-can},
	colorlinks = true,
	citecolor = black,
	filecolor = black,
	linkcolor = black,
	urlcolor = black,
	linktoc = all,
}

\title{\thesisTitle}
\author{\thesisAuthor}
\date{\today{}}

\begin{document}
	% Vorspann einleiten:
	\frontmatter
	
	\begin{titlepage}
	
\begin{center}
		\textbf{\Large Entwurf und Realisierung eines}\\
		\textbf{\Large Capture the Flag Core Systems}\\[3cm]
		\textbf{Bachelorarbeit}\\
		zur Erlangung des Grades {\em Bachelor of Science}\\[1.5cm]
		
		an der\\
		Hochschule Niederrhein\\
		Fachbereich Elektrotechnik und Informatik\\
		Studiengang {\em Informatik}\\[3cm]
		
		vorgelegt von\\
		\thesisAuthor\\
		Matrikelnummer: \Matrikelnummer\\[3cm]
		Datum: 4. August 2020\\[3cm]
		
		Prüfer:~Prof.~Dr.~Jürgen Quade\\
		Zweitprüfer:~Prof.~Dr.~Peter~Davids
	\end{center}
\end{titlepage}
	\cleardoublepage
	\pagestyle{scrheadings}
	%-------------------------------------
\section*{Eidesstattliche Erklärung}
%-------------------------------------

\begin{tabbing}
	Matrikelnummer: \= \kill
	Name: \> \thesisAuthor\\
	Matrikelnr.: \> \Matrikelnummer\\
	Titel: \> \thesisTitle\\
	Englischer Titel: \> \thesisTitleEnglish\\
\end{tabbing}

Ich versichere durch meine Unterschrift, dass die vorliegende Arbeit ausschließlich von mir verfasst wurde.
Es wurden keine anderen als die von mir angegebenen Quellen und Hilfsmittel benutzt.

Die Arbeit besteht aus \underline{\hspace{3em}} Seiten.

\vspace{6ex}
\begin{tabbing}
\underline{\hspace{14em}} \hspace{3em}\= \underline{\hspace{14em}} \\
Ort, Datum \> \thesisAuthor
\end{tabbing}
	\tableofcontents	

	
	% Hauptteil einleiten:
	\mainmatter
	
	% Einstellungen für die Fußzeile aktualisieren
		% Einstellungen nur für die rechten Seiten
		% Layout: Abschnitt  |  Seite
	\rofoot{\textbf{\headmark~~$\mid$~~\pagemark}}
		% Einstellungen nur für die linken Seiten
		% Layout: Seite  |  Kapitel
	\lefoot{\textbf{\pagemark~~$\mid$~~Kapitel~\headmark}}
	
	\chapter{Einleitung}
	\section{Motivation}
\label{sec:Motivation}
Neben diversen Meldungen zu erfolgreichen Angriffen auf Unternehmen und öffentliche Körperschaften und durch die Veranstaltung IT-Sicherheit im fünften Semester, besonders herauszuheben ist hier das Praktikum\footnote{Praktikum ist hierbei mit einer Pflichtübung vergleichbar}, bin ich auf das Thema IT-Sicherheit aufmerksam geworden. 

Die zunehmenden Vorfälle zeigen, dass ein breites Bewusstsein für IT-Sicherheit geschaffen werden muss.

Der Versuch \textquote{Catch me, if you can} versucht dieses Bewusstsein zu schaffen, in dem die Studierenden sowohl in die Rolle des Angreifers als auch die des Schützers schlüpfen.

Das Programm, welches das Praktikum überwacht, ist bereits 10 Jahre alt und bietet meiner Meinung nach Notwendigkeiten der Modernisierung, Überarbeitung und Erweiterung. 
So gibt es beispielsweise heute bessere Möglichkeiten die Darstellung (Web-Oberfläche) zu realisieren.
	%\input{20-Einleitung/xyz}
	
	\chapter{Analyse}
	%\input{30-Analyse/xyz}
	
	\chapter{Entwurf}
	%\input{40-Entwurf/xyz}
	
	\chapter{Realisierung}
	%\input{50-Realisierung/xyz}
	
	\chapter{Ergebnis}
	%\input{60-Ergebnis/xyz}
	
	\chapter{Zusammenfassung \& Aussicht}
	%\input{70-Ende/xyz}
	
	% Nachspann einleiten:
	\backmatter 
	% Einstellungen für die Fußzeile aktualisieren
	% 	Einstellungen nur für die rechten Seiten
	% 	Layout: Abschnitt  |  Seite
	\rofoot{\textbf{$\mid$~~\pagemark}}
	%	Einstellungen nur für die linken Seiten
	%	Layout: Seite  |  Kapitel
	\lefoot{\textbf{\pagemark~~$\mid$}}
	\chapter{Anhang}
	%\input{90-Anhang/xyz}
	
	% Tabellenverzeichnis
	\listoffigures
	
	% Abbildungsverzeichnis
	\listoftables
	
	% Quellenverzeichnis
	\printbibliography
	 
	% Stichwortverzeichnis
	\printindex
	
\end{document}