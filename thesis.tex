\documentclass[a4paper, 12pt, parskip=half]{scrbook}

% Einfaches Copy and Paste ermöglichen
\usepackage{cmap}

% deutsche Silbentrennung
\usepackage[ngerman]{babel}

% deutsche Umlaute
\usepackage[utf8]{inputenc}
\usepackage[T1]{fontenc}

\usepackage{csquotes}

% Keine extra Leerzeichen nach einem Punkt
\frenchspacing

% Schriftart
\usepackage{times}

% Stichwortverzeichnis
\usepackage{imakeidx}
\makeindex

% Paket für Änderungen an Kopf- und Fußzeile
\usepackage{scrlayer-scrpage}
% Bisherige Einstellungen für Kopf- und Fußzeilen löschen:
\clearpairofpagestyles
% Einstellungen für die Fußzeile, die Kopfzeile wird nicht verwendet
	% Einstellungen nur für die rechten Seiten
	% Layout: |  Seite
\rofoot[\textbf{$\mid$~~\pagemark}]{\textbf{$\mid$~~\pagemark}}
	% Einstellungen nur für die linken Seiten
	% Layout: Seite  |
\lefoot[\textbf{\pagemark~~$\mid$}]{\textbf{\pagemark~~$\mid$}}

% Paket für code Beispiele
% \usepackage{listings}

% Paket für Komma getrennte Fußnoten
\usepackage[multiple]{footmisc}

% Paket für Verlinkungen
\PassOptionsToPackage{hyphens}{url}\usepackage[breaklinks]{hyperref}

\renewcommand{\UrlBreaks}{\do\/\do\-\do\_\do\&\do\?}	% allows URL breaking on /, -, _ and &

% Eigene Commands
\newcommand{\thesisTitle}{Vorläufiges Thema}
\newcommand{\thesisSubject}{Vorläufiges Subject}
\newcommand{\thesisAuthor}{Robert Hartings}
\newcommand{\Matrikelnummer}{1164453}


% Literaturverzeichnis einrichten
\usepackage[style=ieee, citestyle=ieee]{biblatex}
\addbibresource{thesis.bib}

\hypersetup{
	unicode = true, % allows to use characters of non-Latin based languages in Acrobat’s bookmarks 
	pdftitle = {\thesisTitle}, % define the title that gets displayed in the "Document Info" window of Acrobat 
	pdfauthor = {\thesisAuthor},
	pdfsubject = {\thesisSubject},
	pdfkeywords = {ITS2, hack-me-if-you-can},
	colorlinks = true,
	citecolor = black,
	filecolor = black,
	linkcolor = black,
	urlcolor = black,
	linktoc = all,
}

\title{\thesisTitle}
\author{\thesisAuthor}
\date{\today{}}

\begin{document}
	% Vorspann einleiten:
	\frontmatter
	
	\begin{titlepage}
	
\begin{center}
		\textbf{\Large Entwurf und Realisierung eines}\\
		\textbf{\Large Capture the Flag Core Systems}\\[3cm]
		\textbf{Bachelorarbeit}\\
		zur Erlangung des Grades {\em Bachelor of Science}\\[1.5cm]
		
		an der\\
		Hochschule Niederrhein\\
		Fachbereich Elektrotechnik und Informatik\\
		Studiengang {\em Informatik}\\[3cm]
		
		vorgelegt von\\
		\thesisAuthor\\
		Matrikelnummer: \Matrikelnummer\\[3cm]
		Datum: 4. August 2020\\[3cm]
		
		Prüfer:~Prof.~Dr.~Jürgen Quade\\
		Zweitprüfer:~Prof.~Dr.~Peter~Davids
	\end{center}
\end{titlepage}
	\cleardoublepage
	\pagestyle{scrheadings}
	%-------------------------------------
\section*{Eidesstattliche Erklärung}
%-------------------------------------

\begin{tabbing}
	Matrikelnummer: \= \kill
	Name: \> \thesisAuthor\\
	Matrikelnr.: \> \Matrikelnummer\\
	Titel: \> \thesisTitle\\
	Englischer Titel: \> \thesisTitleEnglish\\
\end{tabbing}

Ich versichere durch meine Unterschrift, dass die vorliegende Arbeit ausschließlich von mir verfasst wurde.
Es wurden keine anderen als die von mir angegebenen Quellen und Hilfsmittel benutzt.

Die Arbeit besteht aus \underline{\hspace{3em}} Seiten.

\vspace{6ex}
\begin{tabbing}
\underline{\hspace{14em}} \hspace{3em}\= \underline{\hspace{14em}} \\
Ort, Datum \> \thesisAuthor
\end{tabbing}
	\tableofcontents	

	
	% Hauptteil einleiten:
	\mainmatter
	
	% Einstellungen für die Fußzeile aktualisieren
		% Einstellungen nur für die rechten Seiten
		% Layout: Abschnitt  |  Seite
	\rofoot[\textbf{\headmark~~$\mid$~~\pagemark}]{\textbf{\headmark~~$\mid$~~\pagemark}}
		% Einstellungen nur für die linken Seiten
		% Layout: Seite  |  Kapitel
	\lefoot[\textbf{\pagemark~~$\mid$~~Kapitel~\headmark}]{\textbf{\pagemark~~$\mid$~~Kapitel~\headmark}}
	
	\chapter{Einleitung}
	\label{chap_text:Einleitung}
Das Thema IT Sicherheit ist besonders in den letzten Jahren relevant geworden. Viele Firmen suchen Fachkundige \cite{it-daily.netITSecurityExpertenWerdenHanderingend2019}, die die bestehenden und neu designten Systeme und Programme auf Sicherheitslücken prüfen und Lösungsvorschläge zu deren Behebung präsentieren. Auch werden Experten gesucht, die die im Unternehmen bestehenden Prozesse prüfen und neue zum Umgang mit Sicherheitslücken entwerfen.

Einen Mangel an IT-Sicherheit in privaten und öffentlichen Unternehmen, beziehungsweise ein fehlendes Konzept zur Vorbeugung, Erkennung und Abwendung von Sicherheitslücken sieht man in jüngster Vergangenheit deutlich, nachdem beispielsweise diverse Universitäten wie Gießen \cite{schirmacherUniGiessenNaehert2020}, Maastricht \cite{wdrCyberattackeHackerangriffAuf2019} und Bochum \cite{ruhr24HackerAngriffLegtITSysteme2020} Ende 2019 Ziele von Hackerangriffen geworden sind. Aber nicht nur Universitäten sind betroffen, so sind neben Gerichten \cite{hurtzHackerAngriffAufGericht2020}, Stadtverwaltungen \cite{barsigCyberAttackeAufPotsdamer2020} und Krankenhäusern \cite{wellbrockITSicherheitImKrankenhaus2019} bereits der Deutsche Bundestag von Hackern angegriffen und kompromittiert worden. \cite{fladeCyberangriffAufBundestag2020}

In der Studie \textquote{Wirtschaftsschutz in der digitalen Welt} vom 06. November 2019 des Bundesverbandes Informationswirtschaft, Telekommunikation und neue Medien e.V. Bitkom wird die aktuelle Bedrohungslage durch Spionage und Sabotage für deutsche Unternehmen untersucht. Daraus geht hervor, dass im Jahr 2019 von Datendiebstahl, Industriespionage oder Sabotage 75\% der befragten Unternehmen\footnote{Die Grundlage der Studie sind 1070 (2019) und 1074 (2015) befragte Unternehmen} betroffen  und 13\% vermutlich betroffen waren. Die Zahlen dieser Unternehmen ist steigend. Im Jahre 2015 waren \textquote{nur} 51\% betroffen und 28\% vermutlich betroffen. Die Unternehmen beziffern den Schaden auf 102,9 Milliarden Euro pro Jahr. \cite{bergWirtschaftsschutzDigitalenWelt2019}

Dass diese Thematik auch im Lehrbetrieb angekommen ist, sieht man an neu startenden Studiengängen wie dem Bachelorstudiengang Cyber Security Management der Hochschule Niederrhein, der zum Wintersemester 2020/21 startet. \cite{hochschuleniederrheinHackernRoteKarte2020}

Es ist zu erwähnen, dass die Hochschulen sich bereits mit dem Thema auseinandersetzen. 
So beschäftigt sich an der Hochschule Niederrhein das Institut für Informationssicherheit Clavis besonders mit Themen rund um das Informationssicherheitsmanagement, gestaltet aber auch Inhalte zur Vulnerabilität von (kritischer) Infrastruktur und Hacking.
Das Ziel von Clavis ist die Erhöhung der Informationssicherheit von Organisationen im regionalen Umfeld der Hochschule.
\cite{hochschuleniederrheinFlyerInstitutClavis}
Auch hat die Hochschule Niederrhein das Thema IT-Sicherheit bereits in ihren Lehrplan für die Studiengänge Informatik und Elektrotechnik am Fachbereich 03 Elektrotechnik und Informatik aufgenommen. So werden dort im fünften Semester in der Veranstaltung IT-Security grundlegende Kompetenzen zum Thema IT-Sicherheit vermittelt, welche einem allgemeinen Anspruch genügen. \cite{hochschuleniederrheinModulhandbuchVollzeitBA2019}

Begleitend zu dieser Veranstaltung werden drei Versuche im Rahmen des Praktikums durchgeführt. Diese sollen den Studierenden praktische Erfahrungen ermöglichen und ein breites Bewusstsein schaffen, indem die Studierenden sowohl in die Rolle des Angreifers als auch die des Verteidigers schlüpfen.

Im zweiten Versuch namens \textquote{Catch me, if you can} findet ein Vergleichswettbewerb zwischen den teilnehmenden Studierendenteams statt.

Die Aufgabe der Teams besteht darin, festgelegte Programme/Dienste abgesichert bereit zustellen, geheime Informationen sowohl auf dem eigenen Rechner als auch auf den Rechnern der anderen Teams zu finden und Schwachstellen abzusichern, um so zu verhindern, dass andere Teams an die eigenen geheimen Informationen gelangen. \cite[S. 2]{sosnaKonzeptionUndRealisierung2010} 

Die geheimen Informationen sind normalerweise Passwörter oder private Bilder und werden im Versuch durch sogenannte Flags repräsentiert. Eine Flag ist eine generierte Zeichenfolge mit fester Länge.
	\section{Motivation}
\label{sec:Motivation}
Neben diversen Meldungen zu erfolgreichen Angriffen auf Unternehmen und öffentliche Körperschaften und durch die Veranstaltung IT-Sicherheit im fünften Semester, besonders herauszuheben ist hier das Praktikum\footnote{Praktikum ist hierbei mit einer Pflichtübung vergleichbar}, bin ich auf das Thema IT-Sicherheit aufmerksam geworden. 

Die zunehmenden Vorfälle zeigen, dass ein breites Bewusstsein für IT-Sicherheit geschaffen werden muss.

Der Versuch \textquote{Catch me, if you can} versucht dieses Bewusstsein zu schaffen, in dem die Studierenden sowohl in die Rolle des Angreifers als auch die des Schützers schlüpfen.

Das Programm, welches das Praktikum überwacht, ist bereits 10 Jahre alt und bietet meiner Meinung nach Notwendigkeiten der Modernisierung, Überarbeitung und Erweiterung. 
So gibt es beispielsweise heute bessere Möglichkeiten die Darstellung (Web-Oberfläche) zu realisieren.
	\section{Aufgabenstellung}
\label{sec:Aufgabenstellung}
Begleitend zu der Veranstaltung IT-Security für die Studiengänge Bachelor Informatik und den Bachelor Elektrotechnik des Fachbereichs 03 Elektrotechnik und Informatik der Hochschule Niederrhein werden 3 Praktika durchgeführt. Diese sollen den Studierenden praktisch Erfahrungen ermöglichen.

Das zweite Praktikum \textquote{Catch me, if you can} stellt einen Vergleichswettbewerb dar. An diesem Wettbewerb nehmen mehrere Teams teil, welche sich alle in einem gemeinsamen Netzwerk befinden. Die Aufgabe der Teams besteht darin, festgelegte IT-Dienste (abgesichert) bereit zustellen, geheime Informationen sowohl auf dem eigenen Rechner als auch auf den Rechnern der anderen Teams zu finden und zu verhindern, dass andere Teams an die eigenen geheimen Informationen gelangen.\cite[S. 2]{sosnaKonzeptionUndRealisierung2010} Die geheimen Informationen sind logisch gesehen Passwörter oder private Bilder und werden durch sogenannte Flags repräsentiert. Eine Flag ist ein gehashter Zeichenfolge und hat immer die gleiche Länge.

Das Praktikum wird durch ein Auswertungs- und Überwachungssystem überwacht - anderes Wort -, welches eine objektiv nachvollziehbare Bewertung vornehmen kann und die in den Bewertungsprozess eingeflossenen Parameter dokumentiert.\cite[S. 2]{sosnaKonzeptionUndRealisierung2010}

Ziel meiner Arbeit ist die Modernisierung und Verbesserung dieses Auswertungs- und Überwachungssystems.

In der einführenden Betrachtung (\autoref{chap:Analyse}) wird der aktuelle Stand des Systems, Schnittstellen zwischen Server und Client sowie der Begründung für die Veränderung dargelegt. 

Aus dieser einführenden Betrachtung werden dann im \autoref{chap:Entwurf} Entwurf Anforderungen abgeleitet und Entwürfe für die verschieden Komponenten des Servers erstellt. 

An Hand der abgeleiteten Anforderungen und des Entwurfs der verschiedenen Komponenten wird im \autoref{chap:Technologien} Technologien verschiedene Technologien diskutiert und passende Technologien ausgewählt.

Die Implementierung des Entwurfs mit den gewählten Technologien wird im \autoref{chap:Realisierung} Realisierung beschrieben.

Eine kritische Auseinandersetzung mit dem Ergebnis dieser Arbeit folgt und es werden Aussichten für mögliche Veränderungen und Verbesserungen gegeben.
	
	\chapter{Analyse}
	%\input{30-Analyse/xyz}
	
	\chapter{Entwurf}
	%\input{40-Entwurf/xyz}
	
	\chapter{Realisierung}
	%\input{50-Realisierung/xyz}
	
	\chapter{Ergebnis}
	%\input{60-Ergebnis/xyz}
	
	\chapter{Zusammenfassung \& Aussicht}
	%\input{70-Ende/xyz}
	
	% Nachspann einleiten:
	\backmatter 
	% Einstellungen für die Fußzeile aktualisieren
	% 	Einstellungen nur für die rechten Seiten
	% 	Layout: Abschnitt  |  Seite
	\rofoot[\textbf{$\mid$~~\pagemark}]{\textbf{$\mid$~~\pagemark}}
	%	Einstellungen nur für die linken Seiten
	%	Layout: Seite  |  Kapitel
	\lefoot[\textbf{\pagemark~~$\mid$}]{\textbf{\pagemark~~$\mid$}}
	\chapter{Anhang}
	%\input{90-Anhang/xyz}
	
	% Tabellenverzeichnis
	\listoffigures
	
	% Abbildungsverzeichnis
	\listoftables
	
	% Quellenverzeichnis
	\printbibliography
	 
	% Stichwortverzeichnis
	\printindex
	
\end{document}