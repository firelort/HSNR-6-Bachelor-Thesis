\documentclass[a4paper, 12pt, parskip=half]{scrbook}

% Einfaches Copy and Paste ermöglichen
\usepackage{cmap}

% deutsche Silbentrennung
\usepackage[ngerman]{babel}

% deutsche Umlaute
\usepackage[utf8]{inputenc}
\usepackage[T1]{fontenc}

% Keine extra Leerzeichen nach einem Punkt
\frenchspacing

% Schriftart
\usepackage{times}

% Stichwortverzeichnis
\usepackage{imakeidx}
\makeindex

% Paket für Änderungen an Kopf- und Fußzeile
\usepackage{scrlayer-scrpage}
% Bisherige Einstellungen für Kopf- und Fußzeilen löschen:
\clearpairofpagestyles
% Einstellungen für die Fußzeile, die Kopfzeile wird nicht verwendet
	% Einstellungen nur für die rechten Seiten
	% Layout: |  Seite
\rofoot{\textbf{$\mid$~~\pagemark}}
	% Einstellungen nur für die linken Seiten
	% Layout: Seite  |
\lefoot{\textbf{\pagemark~~$\mid$}}

% Paket für code Beispiele
% \usepackage{listings}

% Paket für Verlinkungen
\usepackage{url, hyperref}

\renewcommand{\UrlBreaks}{\do\/\do\-\do\_}	% allows URL breaking on /, - and _
\usepackage{breakurl}

% Eigene Commands
\newcommand{\thesisTitle}{ASDF}
\newcommand{\thesisSubject}{ASDF2}
\newcommand{\thesisAuthor}{Robert Hartings}
\newcommand{\Matrikelnummer}{1164453}


% Literaturverzeichnis einrichten
\usepackage[style=alphabetic, citestyle=alphabetic]{biblatex}
\addbibresource{thesis.bib}

\hypersetup{
	unicode = true, % allows to use characters of non-Latin based languages in Acrobat’s bookmarks 
	pdftitle = {\thesisTitle}, % define the title that gets displayed in the "Document Info" window of Acrobat 
	pdfauthor = {\thesisAuthor},
	pdfsubject = {\thesisSubject},
	pdfkeywords = {ITS2, hack-me-if-you-can},
	colorlinks = true,
	citecolor = black,
	filecolor = black,
	linkcolor = black,
	urlcolor = black,
	linktoc = all,
}

\title{\thesisTitle}
\author{\thesisAuthor}
\date{\today{}}

\begin{document}
	% Vorspann einleiten:
	\frontmatter
	
	\begin{titlepage}
	
\begin{center}
		\textbf{\Large Entwurf und Realisierung eines}\\
		\textbf{\Large Capture the Flag Core Systems}\\[3cm]
		\textbf{Bachelorarbeit}\\
		zur Erlangung des Grades {\em Bachelor of Science}\\[1.5cm]
		
		an der\\
		Hochschule Niederrhein\\
		Fachbereich Elektrotechnik und Informatik\\
		Studiengang {\em Informatik}\\[3cm]
		
		vorgelegt von\\
		\thesisAuthor\\
		Matrikelnummer: \Matrikelnummer\\[3cm]
		Datum: 4. August 2020\\[3cm]
		
		Prüfer:~Prof.~Dr.~Jürgen Quade\\
		Zweitprüfer:~Prof.~Dr.~Peter~Davids
	\end{center}
\end{titlepage}
	\cleardoublepage
	\pagestyle{scrheadings}
	%-------------------------------------
\section*{Eidesstattliche Erklärung}
%-------------------------------------

\begin{tabbing}
	Matrikelnummer: \= \kill
	Name: \> \thesisAuthor\\
	Matrikelnr.: \> \Matrikelnummer\\
	Titel: \> \thesisTitle\\
	Englischer Titel: \> \thesisTitleEnglish\\
\end{tabbing}

Ich versichere durch meine Unterschrift, dass die vorliegende Arbeit ausschließlich von mir verfasst wurde.
Es wurden keine anderen als die von mir angegebenen Quellen und Hilfsmittel benutzt.

Die Arbeit besteht aus \underline{\hspace{3em}} Seiten.

\vspace{6ex}
\begin{tabbing}
\underline{\hspace{14em}} \hspace{3em}\= \underline{\hspace{14em}} \\
Ort, Datum \> \thesisAuthor
\end{tabbing}
	\tableofcontents	

	
	% Hauptteil einleiten:
	\mainmatter
	
	% Einstellungen für die Fußzeile aktualisieren
		% Einstellungen nur für die rechten Seiten
		% Layout: Abschnitt  |  Seite
	\rofoot{\textbf{\headmark~~$\mid$~~\pagemark}}
		% Einstellungen nur für die linken Seiten
		% Layout: Seite  |  Kapitel
	\lefoot{\textbf{\pagemark~~$\mid$~~Kapitel~\headmark}}
	
	\chapter{Einleitung}
	%\input{20-Einleitung/xyz}
	
	\chapter{Analyse}
	%\input{30-Analyse/xyz}
	
	\chapter{Entwurf}
	%\input{40-Entwurf/xyz}
	
	\chapter{Realisierung}
	%\input{50-Realisierung/xyz}
	
	\chapter{Ergebnis}
	%\input{60-Ergebnis/xyz}
	
	\chapter{Zusammenfassung \& Aussicht}
	%\input{70-Ende/xyz}
	
	% Nachspann einleiten:
	\backmatter 
	% Einstellungen für die Fußzeile aktualisieren
	% 	Einstellungen nur für die rechten Seiten
	% 	Layout: Abschnitt  |  Seite
	\rofoot{\textbf{$\mid$~~\pagemark}}
	%	Einstellungen nur für die linken Seiten
	%	Layout: Seite  |  Kapitel
	\lefoot{\textbf{\pagemark~~$\mid$}}
	\chapter{Anhang}
	%\input{90-Anhang/xyz}
	
	% Tabellenverzeichnis
	\listoffigures
	
	% Abbildungsverzeichnis
	\listoftables
	
	% Quellenverzeichnis
	\printbibliography
	 
	% Stichwortverzeichnis
	\printindex
	
\end{document}