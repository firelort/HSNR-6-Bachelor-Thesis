\section{Bedienungsanleitung}

Das Game Information System ist ein RESTful-Interface und kann daher mit einem REST-Client angesprochen werden. Dazu kann die entworfene SPA implementiert werden. Es können aber auch Programme wie cURL -- \textquote[\cite{curlCurlHowUse}]{Werkzeug zur Übertragung von Daten von oder zu einem Server} -- oder Insomnia\footnote{\url{https://insomnia.rest/}} verwendet werden. Insomnia wurde während der Entwicklung eingesetzt.

Über einen REST-Client können, alle implementierten Routen (\autoref{table:gis-routes}) angesprochen werden. Für die Authentifizierung muss ein Access-Token über die Login-Schnittstelle abgeholt werden und in den nächsten Anfragen mitgesendet werden.

In der Dokumentation des GIS wird unter anderem auf die implementierten Routen eingegangen. Dort und in \autoref{table:gis-permission-1}/\ref{table:gis-permission-2} kann eingesehen werden, ob eine Authentifizierung notwendig ist. Welche Daten mitgesendet werden müssen und wie die Antwort aufgebaut ist, kann der Dokumentation entnommen werden.