\section{Big Brother}

Im Folgenden wird die Implementierung des Scanners namens Big Brother beschrieben.

Für die Realisierung wurde die Programmiersprache Python gewählt, da diese bereits im \linebreak Game Information System eingesetzt wird und zum anderen in der Lehre der Hochschule Niederrhein vorkommt. Eine Programmierung in beispielsweise C oder C++ wäre auch denkbar gewesen, wurde aber aufgrund der unterschiedlichen Syntaxen zwischen C und Python nicht gewählt.

Big Brother baut zu Beginn eine Verbindung mithilfe der in der Konfiguration (\autoref{lst:anhang-bigbrother-config-example}) hinterlegten Informationen zur Datenbank auf. In der Datenbank befinden sich die Daten, die von Big Brother und dem GIS gemeinsam genutzt werden. Diese sind beispielsweise für Big Brother die auszuführende Aufgabe und die zu überwachenden GameClients und für das GIS die Ergebnisse der Scans.

Um die Verbindung zur PostgreSQL Datenbank aufzubauen, wird das Python-Paket \textit{psycopg2} verwendet. Dieses dient als Adapter zwischen PostgreSQL Datenbanken und \linebreak Python-Programmen. 

Nach dem erfolgreichen Aufbau der Verbindung zur Datenbank wird die Aufgabe des \linebreak Scanners bestimmt. Sollte in der Datenbank nicht explizit die Aufgabe der Initialisierung \linebreak spezifiziert werden, wird angenommen, dass die GameClients überwacht werden sollen. (siehe \autoref{lst:bigbrother-startup})

\begin{lstlisting}[language=Python, frame=single, caption={Aufgabe des Scanners}, captionpos=b, label={lst:bigbrother-startup}]
cursor.execute(f"SELECT value FROM {SETTING_TABLE_NAME} WHERE key = 'scanner.task'")
task_tuple = cursor.fetchone()
	
try:
	task = task_tuple[0].upper()
except TypeError:
	task = "SCAN"
...
if task != "INIT":
	start_scan(connection)
else:
	run_init(connection)	
\end{lstlisting}


\subsection{Implementierung der Initialisierung}

Bei der Initialisierung wird zuerst die bestehende Datenbanktabelle mit den vorhanden \linebreak Services zurückgesetzt. Im Anschluss werden alle im Big Brother implementierten Services in die Datenbanktabelle geschrieben. Dazu wird ein Dummy-Scanner angelegt und die Scan-Operationen ausgelesen.

Durch diese Vorgehensweise können alte Datenbestände keine Fehler verursachen.

\begin{lstlisting}[language=Python, frame=single, caption={Löschen der Services}, captionpos=b, label={lst:bigbrother-delete-service}]
cursor.execute(f"DELETE FROM {SERVICE_TABLE_NAME}")	
\end{lstlisting}

Bei dem Einfügen der Services werden pro Service der interne und der angezeigte Name sowie die Gewichtung und die Aktivierung in die Tabelle (\textit{services}) geschrieben. Bei der Gewichtung und der Aktivierung werden Standardwerte verwendet, damit dieser Vorgang automatisch durchgeführt werden kann. Standardmäßig wird jeder Service mit 1 gewichtet und ist deaktiviert.

\begin{lstlisting}[language=Python, frame=single, caption={Einfügen eines Services}, captionpos=b, label={lst:bigbrother-create-service}]
cursor.execute(f"INSERT INTO {SERVICE_TABLE_NAME}(name, public_name, weight, active) VALUES ('{name}', '{public_name}', 1, FALSE);")
\end{lstlisting}

Die Standardwerte können pro Service über das GIS überschrieben werden.

Außerdem werden Standardwerte für Einstellungen in die Datenbanktabelle (\textit{settings}) geschrieben, sofern diese nicht bereits vorhanden sind.

\begin{lstlisting}[language=Python, frame=single, caption={Einfügen einer Einstellung}, captionpos=b, label={lst:bigbrother-create-settings}]
cursor.execute(f"INSERT INTO {SETTING_TABLE_NAME}(key, value) VALUES ('{default_value[0]}', '{default_value[1]}') ON CONFLICT (key) DO NOTHING;")
\end{lstlisting}


\subsection{Implementierung der Scan-Operations}
Die im alten System vorhandene Überwachung ist in der Programmiersprache PHP implementiert und wurde im Rahmen dieser Arbeit in das Python-Programm überführt.

Die verschiedenen Aufgaben der Überwachung wurden, wie im Entwurf ausgearbeitet, in einzelne sogenannte Scan-Operationen getrennt. Dies soll das Verständnis fördern und den Aufwand einer Wartung reduzieren. Durch diesen Ansatz können einzelne Scan-Operationen einfacher ersetzt oder verändert werden. Außerdem ist es so möglich, leichter neue Scan-Operationen zu integrieren.

\subsubsection{Basisklasse der Scan-Operationen}

Alle implementierten Klassen werden von der Basisklasse \textit{AbstractScanOperation} abgeleitet. Diese selbst wird von der in Python implementierten abstrakten Basisklasse \textit{ABC} abgeleitet.

In der Basisklasse \textit{AbstractScanOperation} wird die Funktion \textit{\_is\_port\_open} implementiert. Mithilfe dieser kann geprüft werden, ob ein bestimmter Port auf einem Rechner geöffnet ist. Hierzu wird mit der Bibliothek \textit{socket} eine Socket-Verbindung zum Rechner gestartet. Schlägt die Verbindung fehl, wird angenommen, dass der Port auf dem Rechner geschlossen ist.

\begin{lstlisting}[language=Python, frame=single, caption={Big Brother Funktion is\_port\_open}, captionpos=b, label={lst:bigbrother-port-open}]
try:
	socket_connection = socket.socket(socket.AF_INET, socket.SOCK_STREAM)
	socket_connection.connect((self._ip, port))
	socket_connection.shutdown(socket.SHUT_RDWR)
	return True
except ...
	return False
finally:
	if socket_connection is not None:
		socket_connection.close()
\end{lstlisting}

Die Funktion wird in der Basisklasse zur Verfügung gestellt, da verschiedene Scan-Operationen diese benötigen und nutzen.

Darüber hinaus definiert die Basisklasse die abstrakte Funktion \textit{start}, die von allen abgeleiteten Scan-Operationen implementiert wird. Dadurch ist es möglich, die Scan-Operationen einheitlich zu starten.

Die Ergebnisse der Scan-Operationen werden ebenfalls einheitlich im privaten Attribut \textit{\_result} abgespeichert. Das Ergebnis kann über das getter-Property \textit{result} abgerufen werden. Zum Abruf des internen und des Anzeigenamens werden ebenfalls getter-Properties eingesetzt. 

Durch die Nutzung einer Art getter-Methode wird anderen Entwickelnden mitgeteilt, dass keine Änderungen an dem privaten Attribut durchgeführt werden sollen. Mithilfe des in Python vorhandenen Property Decorators kann die getter-Methode als Variable und nicht als Funktion verwendet werden.

\begin{lstlisting}[language=Python, frame=single, caption={Big Brother Ergebnis Getter-Property}, captionpos=b, label={lst:bigbrother-getter-result}]
@property
def result(self) -> bool:
	return self._result
\end{lstlisting}

\subsubsection{Host}
In der HostUp-Operation wird geprüft, ob der entfernte Rechner erreichbar ist. Hierfür wird ein einfaches ICMP-Paket (Ping) gesendet. Wird eine Antwort auf das ICMP-Paket empfangen, ist der entfernte Rechner erreichbar.

Hierfür wird die Python Bibliothek \textit{ping3} genutzt. Diese stellt ähnlich der \textit{\_is\_port\_open} eine Socketverbindung her. Die Besonderheit ist, dass statt einem TCP-Paket ein ICMP-Paket versendet wird. Da die Erstellung eines ICMP-Paketes nicht ohne Root-Rechte funktioniert, benötigt das Programm eben diese.

\begin{lstlisting}[language=Python, frame=single, caption={Big Brother HostUp Ping}, captionpos=b, label={lst:bigbrother-host-up}]
def start(self) -> None:
result = ping(self._ip, ttl=5, timeout=Config.operations['base']['ping_timeout'])
if result is None:
	self._result = False
else:
	self._result = True
\end{lstlisting}

Für das Versenden eines ICMP-Paketes hätte auch  die in Linux implementierte ping-Funktion per Syscall verwendet werden können. Diese Alternative wurde nicht gewählt, da das Programm unabhängig vom unterliegenden Betriebssystem agieren soll.
 
\subsubsection{Bubble und Bubble-Ng}
Für die beiden Bubble Scan-Operationen wurde eine extra Klasse geschaffen, die eine Methode bereitstellt, mit der die Erreichbarkeit des Bubble-Servers geprüft wird. Diese Klasse wurde geschaffen, da sich die beiden Scan-Operationen nur minimal voneinander unterscheiden.

Die Methode öffnet eine Telnet-Verbindung zum Bubble-Server und führt einen Befehl des Bubble-Servers aus. Sollte die Antwort das erwartete Ergebnis nicht beinhalten oder sollte keine Verbindung aufgebaut werden, wird der Bubble-Server  als nicht erreichbar oder benutzbar angesehen.

\begin{lstlisting}[language=Python, frame=single, caption={Big Brother Buble Port Prüfung}, captionpos=b, label={lst:bigbrother-bubble-up}]
def _is_bubble_port_up(self) -> bool:
  try:
    with Telnet(self._ip, self._port) as connection:
      connection.write(b'help\n')
        if connection.read_until(b"commands:", 1).decode('ascii') == "commands:":
          return True
        else:
          return False
        except ...
          return False
\end{lstlisting} 

Auch implementiert diese Klasse die \textit{start}-Funktion, indem nur die \textit{\_is\_bubble\_port\_up()}-Funktion aufgerufen und das Ergebnis zwischengespeichert wird.
\begin{lstlisting}[language=Python, frame=single, caption={Big Brother Bubble Scan-Operation}, captionpos=b, label={lst:bigbrother-bubble-up-start}]
def start(self) -> None:
	self._result = self._is_bubble_port_up()
\end{lstlisting} 

Die eigentlichen Scan-Operationen \textit{BubbleUp} und \textit{BubbleNgUp} setzen in der Konfiguration der abstrakten Klasse jeweils den zu prüfenden Port. Derzeit läuft auf den GameClients auf Port \textit{12321} der Bubble- und auf Port \textit{12322} der BubbleNg-Server.
\subsubsection{Web}

Die Scan-Operationen \textit{ScanHttpUpOperation} und \textit{ScanHttpsUpOperation} unterscheiden sich nur im zu prüfenden Port. Beide nutzen zur Überprüfung des überwachten Dienstes die von der Basisklasse bereitgestellten Funktion \textit{\_is\_port\_open(PORT)}.

\begin{lstlisting}[language=Python, frame=single, caption={Big Brother HTTP(S) Scan-Operation}, captionpos=b, label={lst:bigbrother-http-up}]
def start(self) -> None:
	result = self._is_port_open(HTTPS_PORT)
	...
	self._result = result
\end{lstlisting} 

Die gegenwärtige Implementierung könnte ähnlich der Bubble Realisierung verändert werden, sodass für die HTTP-Dienste ebenfalls eine abstrakte Klasse definiert wird. Da im Gegensatz aber keine zusätzliche Funktion wie \textit{\_is\_bubble\_port\_up()} benötigt wird, ist von der Realisierung einer allgemeinen HTTP-Klasse abgesehen worden.

\subsubsection{FTP}
In der FTP Scan-Operation soll gepürft werden, ob der FTP-Server nutzbar und ob der anonyme Login abgeschaltet ist. Dazu wird mit der Python Bibliothek \textit{ftplib} eine Verbindung zum FTP-Server ohne Angabe von Nutzerdaten aufgebaut. Ist die Verbindung erfolgreich, wird der Dienst durch die Studierenden nicht abgesichert. Sollte die Verbindung fehlerhaft sein, wird angenommen, dass der Dienst nicht zur Verfügung steht. 

Einzig eine Verbindung, die mit dem Fehler \textquote{anonymous access disabled} beendet wird, zählt als erfolgreich, da hier die Verbindung an fehlenden Nutzerdaten gescheitert ist.
 
\begin{lstlisting}[language=Python, frame=single, caption={Big Brother FTP Scan-Operation}, captionpos=b, label={lst:bigbrother-ftp-save}]
def start(self) -> None:
  try:
    with FTP(self._ip) as connection:
      connection.login()
      self._result = False
    except error_perm as error_msg:
      if "anonymous access disabled" in str(error_msg):
        self._result = True
      else:
        self._result = False
    except ...
      self._result = False
\end{lstlisting} 

\subsubsection{SQL-Injection}

Bei dieser Scan-Operation müssen zwei Dinge geprüft werden. Erstens muss sichergestellt werden, dass die Webseite mit der SQL-Injection erreichbar ist. Zweitens muss geprüft werden, ob die SQL-Injection behoben worden ist. 

Die Erreichbarkeit muss geprüft werden, da ansonsten bei der Prüfung der SQL-Injection fehlerhafte Ergebnisse produziert werden, wenn die Seite nicht erreichbar ist.

Die Erreichbarkeit wird mithilfe einer validen Kombination aus Nutzername und Passwort geprüft. Sollte der Server nicht oder mit einer nicht erwarteten Nachricht antworten, wird die SQL-Injection als nicht nutzbar angesehen. Die Anfrage wird unter Zuhilfenahme der \textit{request}-Bibliothek erstellt und gesendet. 

\begin{lstlisting}[language=Python, frame=single, caption={Big Brother SQL-Injection UP}, captionpos=b, label={lst:bigbrother-sql-injection-up}]
def _is_sql_injection_up(self) -> bool:
 try:
  request_data = {
   "user": Config.operations['sql_injec']['admin']['username'],
   "pass": Config.operations['sql_injec']['admin']['password']
  }

  response = requests.post(f"http://{self._ip}/{URL_PATH}", data=request_data)
  
  if Config.operations['sql_injection']['control']['flag'] in response.text and Config.operations['sql_injection']['control']['value'] in response.text:
   return True
  else:
   return False
 except ...
  return False
\end{lstlisting}

Ob die SQL-Injection abgesichert ist, wird ähnlich wie bei der Nutzbarkeit geprüft. Es werden aber anstatt einer validen Kombination manipulierte Daten gesendet. Sollte nicht die implementierte Fehlermeldung \textquote{Login failed, you n00b!!!} in der Antwort enthalten sein, ist die SQL-Injection nicht ausreichend abgesichert worden.

\begin{lstlisting}[language=Python, frame=single, caption={Big Brother SQL-Injection Save}, captionpos=b, label={lst:bigbrother-sql-injection-save}]
def start(self) -> None:
  if not self._is_sql_injection_up():
    self._result = False
  try:
    request_data = {
      "user": "' OR 1=1  #  ",
      "pass": "not used"
    }

    response = requests.post(f"http://{self._ip}/{URL_PATH}", data=request_data)

    if response.status_code == 500:
      self._result = False
    elif "Login failed, you n00b!!!" in response.text:
      self._result = True
    else:
      self._result = False
  except ...
    self._result = False
\end{lstlisting}

\subsubsection{XSS}
Bei der XSS Scan-Operation wird geprüft, ob das Bewertungsformular auf den GameClients weiterhin für XSS-Angriffe offen ist.

Hierzu werden anstatt einer realen Bewertung der Name und die Bewertung mit JavaScript ersetzt und an den Server gesendet. Wird das gesendete JavaScript ungefiltert in das HTML-Dokument übernommen, wurde die XSS-Schwachstelle nicht behoben.
\begin{lstlisting}[language=Python, frame=single, caption={Big Brother XSS Save}, captionpos=b, label={lst:bigbrother-xss-save}]
def start(self) -> None:
  try:
    request_data = {
      "name": "<script>alert('XSS via name injected');</script>",
      "bewertung": "<script>alert('XSS via bewertung injected');</script>"
    }

    response = requests.post(f"http://{self._ip}/{URL_PATH}", data=request_data)
    
    if request_data['bewertung'] in response.text or request_data['name'] in response.text:
      self._result = False
    else:
      self._result = True
  except ...
    self._result = False
\end{lstlisting}

\subsubsection{Telnet}
Die Klasse der Telnet Scan-Operation implementiert eine Hilfsfunktion, die prüft, ob der Telnetport (\textit{23}) erreichbar ist. Hierzu wird die von der Basisklasse implementierte Methode zum Prüfen offener Ports verwendet.

\begin{lstlisting}[language=Python, frame=single, caption={Big Brother Telnet}, captionpos=b, label={lst:bigbrother-telnet}]
def _is_telnet_up(self) -> bool:
	return self._is_port_open(23)
\end{lstlisting}

Die eigentliche Scan-Operation ruft nur die Hilfsfunktion auf und speichert das invertierte Ergebnis ab, da die Studierenden den Telnet-Server abschalten sollen.

\begin{lstlisting}[language=Python, frame=single, caption={Big Brother Telnet}, captionpos=b, label={lst:bigbrother-telnet-start}]
def start(self) -> None:
	result = self._is_telnet_up()
	self._result = not result
\end{lstlisting}

\subsubsection{Htaccess}

Bei dieser Scan-Operation wird geprüft, ob die auf allen Systemen voreingestellten Nutzerdaten für den Htaccess-Schutz geändert worden sind. Zur Überprüfung wird die geschützte Seite mit den Standardnutzerdaten aufgerufen. Sollte der Statuscode ungleich 401 (Unauthorized) sein, waren die Nutzerdaten korrekt oder der Webserver war nicht erreichbar.
\begin{lstlisting}[language=Python, frame=single, caption={Big Brother Htaccess}, captionpos=b, label={lst:bigbrother-htaccess}]
def start(self) -> None:
  try:
    response = requests.head(f"http://{self._ip}/phpmyadmin/", auth=(Config.operations['htaccess']['username'], Config.operations['htaccess']['password']))
    
    if response.status_code == 401:
      self._result = True
    elif response.status_code ...
      self._result = False
  except ...
    self._result = False
\end{lstlisting}

\subsubsection{SQL-Passwort}

Im Gegensatz zum alten System wird das SQL-Passwort auf dem entfernten Rechner selbst geprüft.
Dazu wird eine SSH-Verbindung mit den Nutzerdaten, die in der \textit{settings}-Tabelle in der Datenbank hinterlegt sind, aufgebaut. Sollte die Verbindung fehlschlagen, wird angenommen, dass das SQL-Passwort nicht geändert worden ist.

Über die SSH Verbindung kann die lokale Datenbank angesprochen werden. Sollte die Authentifizierung mit der Standardkennung aufgrund falscher Nutzerdaten fehlschlagen, wurde das SQL-Passwort geändert.

\begin{lstlisting}[language=Python, frame=single, caption={Big Brother SQL-Passwort}, captionpos=b, label={lst:bigbrother-sql-password}]
def start(self) -> None:
  connection = None
  
  if self._ssh_username is None or self._ssh_username.upper() == "NONE" or self._ssh_password is None or self._ssh_password.upper() == "NONE":
    self._result = False
    return
  
  try:
    connection = paramiko.SSHClient()
    ...
    connection.connect(self._ip, username=self._ssh_username, password=self._ssh_password, timeout=Config.operations['base']['ssh_timeout'])
    
    _, _, stderr = connection.exec_command(f"mysql -u {Config.operations['sql']['username']} -p{Config.operations['sql']['password']} -e 'quit'")
    
    for line in stderr:
      if "Access denied for user 'root'@'localhost' (using password: YES)" in line.strip('\n'):
        self._result = True
        return
    self._result = False
  except ...
    self._result = False
  finally:
    if connection is not None:
      connection.close()
\end{lstlisting}

\subsection{Konfiguration}

\subsubsection{Konstanten}

In der Datei \textit{modules/helper/constant.py} sind Variablen abgelegt, die sich selten bis gar nicht ändern. Diese sind deshalb auch nicht in die Einstellungen übernommen worden.

Zu diesen Variablen zählen die Namen der benutzen Datenbanktabellen und Standardeinstellungen.
Sollte beispielsweise in der Datenbank eine Tabelle umbenannt worden sein, kann diese Änderung in der Datei vollzogen werden. Durch diese Vorgehensweise muss die \linebreak Änderung nicht manuell im nutzenden Code geschehen.

\subsubsection{Konfiguration}

Die Konfiguration in der Datei (\textit{modules/config/config.py}) beinhaltet die Verbindungsdaten zur Datenbank, die URL des GIS -- an die erfolgreiche Scans gemeldet werden sollen -- sowie Einstellungen und Daten für die verschiedenen Scan-Operationen.

Die Standardkonfiguration ist sowohl im Anhang (\autoref{lst:anhang-bigbrother-config-example}) als auch im Big Brother Repository vorhanden.

Einstellungen, die sich meiner Meinung nach häufiger ändern, sind in der Datenbanktabelle \textit{settings} abgelegt, damit diese durch das Hochschulpersonal leichter verändert werden \linebreak können.

\subsection{Implementierung der Scan-Funktion}

Die eigentliche Scan-Funktion wird mithilfe der Klasse \textit{ScanGuard} implementiert, da für jede teilnehmende Gruppe ein Scanner gestartet wird. Bei der Erstellung eines Objektes dieser Klasse werden die Einstellungen, die teilnehmenden Gruppen und die aktiven\linebreak Scan-Operationen übergeben.

Der ScanGuard legt für jede Gruppe ein Objekt der Klasse \textit{Scanner} mit den benötigten Informationen wie IP, aktive Scan-Operationen und Einstellungen an.

Danach wird in einer Endlosschleife der nächste Scan-Zeitpunkt errechnet und im folgenden alle Scanner parallel mit der Bibliothek \textit{concurrent} gestartet. Im Anschluss wird auf die Beendigung der Scanner sowie das Erreichen des nächsten Scan-Zeitpunktes gewartet.

\begin{lstlisting}[language=Python, frame=single, caption={Big Brother ScanGuard}, captionpos=b, label={lst:bigbrother-scanguard}]
while not self._stop_scanning:
  sleep_until = datetime.now() + timedelta(seconds=self._timeout)
  
  future_list = []
  for scanner in scanner_list:
    future_list.append(pool.submit(scanner.start))
    
  wait(future_list)
  ...
  while not self._stop_scanning and datetime.now() < sleep_until:
    sleep(500 / 1000)
\end{lstlisting}

Das Signal für die Beendigung des Programms wird abgefangen, damit die Endlosschleife abgebrochen, und so das Programm gracefull beendet wird.

\subsection{Implementierung eines Scanners}
Bei der Erzeugung eines Objektes der Klassse \textit{Scanner} werden von allen benötigten Scan-Operationen Objekte angelegt.

Wie im Quellcode \autoref{lst:bigbrother-scanner} wird beim Starten eines Scanners anfangs geprüft, ob der entfernte Rechner erreichbar ist. Sollte der entfernte Rechner nicht erreichbar sein, werden keine weiteren Scan-Operationen durchgeführt. Im Anschluss wird noch sequentiell geprüft, ob der HTTP-Dienst erreichbar ist. Ist er erreichbar, werden die Scan-Operationen, die einen funktionierenden HTTP-Dienst voraussetzen, der Liste der durchzuführenden Scan-Operationen hinzugefügt.

Danach werden die restlichen Scan-Operationen parallel gestartet. Nachdem alle \linebreak Scan-Operationen abgeschlossen sind, wird das Ergebnis ausgewertet und in der Datenbank festgehalten. Hierzu wird die Anzahl der durchgeführten Scans und bei Erfolg auch die Anzahl der erfolgreichen Scans um 1 inkrementiert. Außerdem werden der Zeitpunkt des Scans sowie der Erfolg oder Misserfolg vermerkt.

