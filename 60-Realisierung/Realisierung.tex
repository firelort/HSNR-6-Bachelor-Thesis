Die Realisierung ist in drei Teile untergliedert. Zuerst werden die genutzten Datenbanktabellen und -views aufgezeigt. Der zweite Teil beschäftigt sich mit der Umsetzung des Big Brohters. Zuletzt wird die Implementierung des GIS erläutert. Diese Vorgehensweise wurde aufgrund der vorhandenen Abhängigkeiten gewählt.

Der Webclient wurde aus Zeitgründen nicht implementiert und wird daher auch nicht in der Realisierung aufgegriffen.