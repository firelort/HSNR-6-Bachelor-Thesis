\newpage
\section{Ausblick}
\label{sec:Ausblick}
Im Folgenden wird ein Ausblick auf mögliche Veränderungen und Erweiterungen gegeben.

\subsubsection{Implementierung Webseite}
Die angedachte Single Page Applikation konnte aus Zeitgründen nicht in dieser Bachelorarbeit implementiert werden. Die Webseite kann anhand des vorgestellten Entwurfs und der diskutierten Technologie in einer weiteren Bachelorarbeit aufgegriffen werden. Hierbei sollte in jedem Fall ein eigener Vergleich aktueller Technologien sowie eine kritische Auseinandersetzung mit dem in dieser Arbeit erarbeiteten Entwurf für das Frontend durchgeführt werden. Es ist aber auch möglich, die Idee einer SPA zu verwerfen und eine klassische Multi Page Applikation, welche die API verwendet, zu entwickeln.

\subsubsection{Implementierung weiterer Scan-Operationen}
Durch den modularen Aufbau des Scanners ist es leicht möglich, neue Scan-Operationen zu erstellen. Dies kann genutzt werden, um auf dem Client weitere Schwachstellen und/oder Dienste zu implementieren, die dann durch den Big Brother überwacht und durch die Studierenden behoben oder online gehalten werden müssen. Herr Abts hat in seiner Bachelorarbeit\footcite{abtsUeberarbeitungUndErweiterung2016} Ideen für Schwachstellen im Kapitel Ausblick genannt. 

\subsubsection{Veränderungen in der Anwendung}
In der derzeitigen Realisierung wird zum Parsen einer Anfrage der im verwendeten Modul Flask-RESTful implementierte \textit{reqparse (request parser)} verwendet. Dieser soll mit der Flask RESTful Version 2.0.0 entfernt und müsste daher gegen beispielsweise Marshmallow\footnote{\url{https://marshmallow.readthedocs.io/en/stable/}} oder den Flask internen Anfragen-Parser ausgetauscht werden. Es besteht derzeitig keine Dringlichkeit, da Flask-RESTful am 6. Februar 2020 erst in der Version \textit{0.3.8} erschien. Sollte die Anwendung nicht weiterentwickelt werden, ist auch keine Änderung notwendig, da dann die Flask-RESTful Version unverändert bleibt.

\subsubsection{Dummy Client}
Im Versuch werden bei Bedarf GameClients gestartet, die nicht durch die Studierenden geschützt werden. Diese Dummy Clients werden als zusätzliches Angriffsziel benötigt und derzeit gleichwertig behandelt. Eine mögliche Änderung wäre, dass bei dem Dummy Client ein Bool-Wert in der Datenbank bei der Registrierung gesetzt wird. Dieser Wert könnte genutzt werden, um diese GameClients von der Überwachung auszuschließen und in der \linebreak Weboberfläche gesondert darzustellen.

\subsubsection{Tokengenerierung für Flags}
Es ist zu prüfen, ob der Token nicht auf dem Server generiert werden sollte und dem Client nur mitgeteilt wird. Bei der jetzigen Implementierung teilt der GameClient dem Server den Token mit, mit dem die Flags generiert werden sollen. Wenn die Studierenden es schaffen, diesen Token in der Mitteilung an den Server zu verändern, kann dieser auf einen Token gesetzt werden, bei dem die generierten Flags aus alten Versuchen bekannt sind.

Eine Tabelle aller historisch genutzten Tokens kann in der Datenbank angelegt werden, um sicherzustellen, dass jede Gruppe einen historisch einzigartigen Token besitzt. Damit wäre gewährleistet, dass Flags aus alten Versuchsterminen nicht verwendet werden können.

\subsubsection{Lokale Webseiten überarbeiten}
Die auf dem Client betriebenen Webseiten sollten nach der Implementierung der vom Server ausgelieferten Webseite an das Design angepasst werden. Dies erzeugt ein einheitliches Bild und verdeutlicht den Zusammenhang zwischen der lokalen Webseite und dem \linebreak Versuchssystems.

Die Challenge- und Shopseiten sowie die entsprechenden Verweise können auf den lokalen Webseiten entfernt werden, da diese auf der vom Server ausgelieferten Webseite platziert \linebreak werden sollen. Wird von einer Platzierung auf der Server Webseite abgesehen, müssen die \linebreak angesprochen Seiten weiterhin vom GameClient ausgeliefert werden. Damit die bereitgestellte API des GIS verwendet wird, ist eine Änderung der Seiten auf dem GameClient notwendig.

\subsubsection{Flagshop Flags}
Es ist zu überlegen, zu bewerten und zu prüfen, ob eine Abgabe gegnerischer Flagshop Flags sinnvoll ist. Sollte Auswertung einer Sinnhaftigkeit ergeben, muss die implementierte \linebreak Limitierung rückgängig gemacht werden.