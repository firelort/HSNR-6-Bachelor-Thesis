\section{Motivation}
Das Thema IT Sicherheit ist besonders in den letzten Jahren relevant geworden. Viele Firmen suchen Experten\cite{ITSecurityExpertenWerdenHanderingend2019}, welche die bestehen und neue designeten Systeme auf Sicherheitslücken prüfen und Lösungsvorschläge zur deren Behebung präsentieren. Auch werden Experten gesucht, welche die im Unternehmen bestehenden Prozesse prüfen und neue Prozesse zum Umgang mit Sicherheitslücken entwerfen.

Einen Mangel an IT-Security in Unternehmen, privat sowie öffentlich, beziehungsweise ein fehlendes Konzept zur Vorbeugung, Erkennung und Abwendung von Sicherheitslücken sieht man auch in jüngster Vergangenheit deutlich, nachdem beispielsweise diverse Universitäten wie Gießen\footnote{\url{https://www.heise.de/newsticker/meldung/Uni-Giessen-naehert-sich-nach-Hacker-Attacke-wieder-dem-Normalbetrieb-4628715.html}, Abgerufen am 16. Mai 2020}, Maastricht\footnote{\url{https://www1.wdr.de/nachrichten/rheinland/hacker-angriff-uni-maastricht-100.html}, Abgerufen am 16. Mai 2020} und Bochum\footnote{\url{https://www.ruhr24.de/ruhrgebiet/bochum-rub-uni-hacker-angriff-webmail-moodle-news-universitaet-systeme-it-studierende-13753554.html}, Abgerufen am 16. Mai 2020} Ziele von Hackerangriffen geworden sind. Aber nicht nur Universitäten sind betroffen, so ist neben Gerichten\footnote{\url{https://www.sueddeutsche.de/digital/berlin-kammergericht-hacker-angriff-emotet-1.4775305}, Abgerufen am 16. Mai 2020}, Stadtverwaltungen\footnote{\url{https://www.pnn.de/potsdam/hacker-nutzten-sicherheitsluecke-cyber-attacke-auf-das-potsdamer-rathaus/25462398.html}, Abgerufen am 16. Mai 2020}\textsuperscript{,}\footnote{\url{https://www.giessener-allgemeine.de/vogelsbergkreis/alsfeld-wird-erpresst-hacker-angriff-oder-fake-zr-13414725.html}, Abgerufen am 16. Mai 2020} und Krankenhäusern\footnote{\url{https://www.psw-group.de/blog/it-sicherheit-im-krankenhaus-hack-bringt-krankenhaeuser-zum-stillstand/7175}, Abgerufen am 16. Mai 2020} bereits der Deutsche Bundestag\footnote{\url{https://www.tagesschau.de/investigativ/ndr-wdr/hacker-177.html}, Abgerufen am 16. Mai 2020} von Hackern angegriffen und kompromittiert worden.

In der Studie \textquote{Wirtschaftsschutz in der digitalen Welt} vom 06. November 2019 des Bundesverbandes Informationswirtschaft, Telekommunikation und neue Medien e.V. Bitkom wird die aktuelle Bedrohungslage durch Spionage und Sabotage für deutsche Unternehmen untersucht. Aus dieser Studie geht hervor, dass im Jahr 2019 von Datendiebstahl, Industriespionage oder Sabotage 75\% der befragen Unternehmen\footnote{Die Grundlage der Studie für das Jahr 2019 sind 1070 befragte Unternehmen} betroffen  und 13\% vermutlich betroffen waren. Die Unternehmen beziffern den Schaden auf 102,9 Milliarden Euro pro Jahr.\cite{bergWirtschaftsschutzDigitalenWelt2019}

Das dieses auch im Lehrbetrieb angekommen ist, sieht man an neu startenden Studiengängen wie dem Bachelorstudiengang Cyber Security Management der Hochschule Niederrhein, welcher zum kommenden Wintersemester 2020/21 startet.\cite{HackernRoteKarte2020}

Es ist zu erwähnen, dass Hochschulen sich bereits mit dem Thema auseinandersetzen. 
So beschäftigt sich an der Hochschule Niederrhein das Institut für Informationssicherheit Clavis besonders mit Themen rund um das Informationssicherheitsmanagement, gestaltet aber auch Inhalte zur Vulnerabilität von (kritischer) Infrastruktur und Hacking.
Das Ziel von Clavis ist die Erhöhung der Informationssicherheit von Organisationen im regionalen Umfeld der Hochschule.
\cite{FlyerInstitutClavis}
Auch hat die Hochschule Niederrhein das Thema IT-Sicherheit bereits in Ihren Lehrplan für die Studiengänge Informatik und Elektrotechnik am Fachbreich 03 Elektrotechnik und Informatik aufgenommen. So werden dort im 5. Semester in der Veranstaltung IT-Security grundlegenden Kompetenzen zum Thema IT-Sicherheit vermittelt, welche einem allgemeinen Anspruch genügen.\cite{ModulhandbuchVollzeitBA2019}

Durch die Veranstaltung IT-Security im 5. Semester, besonders herauszuheben sind hier die Praktika \footnote{Praktikum ist hierbei mit einer Pflichtübung vergleichbar}, bin ich auf das Thema IT Sicherheit aufmerksam geworden. Da ich sehr viel Spaß am 2. Praktikum \textquote{Catch me, if you can} der Veranstaltungsreihe hatte und ich mich für das Thema Web-Entwicklung / Anwendungsentwicklung interessiere, möchte ich im Rahmen meiner Bachelorarbeit, das mittlerweile 10 Jahre alte System modernisieren und überarbeiten.