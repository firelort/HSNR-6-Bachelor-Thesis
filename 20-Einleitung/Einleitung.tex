\label{chap_text:Einleitung}
Das Thema IT Sicherheit ist besonders in den letzten Jahren relevant geworden. Viele Firmen suchen Experten\cite{it-daily.netITSecurityExpertenWerdenHanderingend2019}, welche die bestehenden und neu designten Systeme auf Sicherheitslücken prüfen und Lösungsvorschläge zur deren Behebung präsentieren. Auch werden Experten gesucht, welche die im Unternehmen bestehenden Prozesse prüfen und neue Prozesse zum Umgang mit Sicherheitslücken entwerfen.

Einen Mangel an IT-Sicherheit in privaten und öffentlichen Unternehmen beziehungsweise ein fehlendes Konzept zur Vorbeugung, Erkennung und Abwendung von Sicherheitslücken sieht man auch in jüngster Vergangenheit deutlich, nachdem beispielsweise diverse Universitäten wie Gießen, Maastricht und Bochum Ende 2019 Ziele von Hackerangriffen geworden sind.\cite{schirmacherUniGiessenNaehert2020}\cite{wdrCyberattackeHackerangriffAuf2019}\cite{ruhr24HackerAngriffLegtITSysteme2020} Aber nicht nur Universitäten sind betroffen, so ist neben Gerichten, Stadtverwaltungen und Krankenhäusern bereits der Deutsche Bundestag von Hackern angegriffen und kompromittiert worden.\cite{hurtzHackerAngriffAufGericht2020}\cite{barsigCyberAttackeAufPotsdamer2020}\cite{wellbrockITSicherheitImKrankenhaus2019}\cite{fladeCyberangriffAufBundestag2020}

In der Studie \textquote{Wirtschaftsschutz in der digitalen Welt} vom 06. November 2019 des Bundesverbandes Informationswirtschaft, Telekommunikation und neue Medien e.V. Bitkom wird die aktuelle Bedrohungslage durch Spionage und Sabotage für deutsche Unternehmen untersucht. Aus dieser Studie geht hervor, dass im Jahr 2019 von Datendiebstahl, Industriespionage oder Sabotage 75\% der befragten Unternehmen\footnote{Die Grundlage der Studie sind 1070 (2019) und 1074 (2015) befragte Unternehmen} betroffen  und 13\% vermutlich betroffen waren. Die Zahlen der betroffenen Unternehmen ist steigend. Im Jahre 2015 waren \textquote{nur} 51\% betroffen und 28\% vermutlich betroffen. Die Unternehmen beziffern den Schaden auf 102,9 Milliarden Euro pro Jahr.\cite{bergWirtschaftsschutzDigitalenWelt2019}

Dass dieser Mangel auch im Lehrbetrieb angekommen ist, sieht man an neu startenden Studiengängen wie dem Bachelorstudiengang Cyber Security Management der Hochschule Niederrhein, welcher zum kommenden Wintersemester 2020/21 startet.\cite{hochschuleniederrheinHackernRoteKarte2020}

Aber es ist zu erwähnen, dass die Hochschulen sich bereits mit dem Thema auseinandersetzen. 
So beschäftigt sich an der Hochschule Niederrhein das Institut für Informationssicherheit Clavis besonders mit Themen rund um das Informationssicherheitsmanagement, gestaltet aber auch Inhalte zur Vulnerabilität von (kritischer) Infrastruktur und Hacking.
Das Ziel von Clavis ist die Erhöhung der Informationssicherheit von Organisationen im regionalen Umfeld der Hochschule.
\cite{hochschuleniederrheinFlyerInstitutClavis}
Auch hat die Hochschule Niederrhein das Thema IT-Sicherheit bereits in Ihren Lehrplan für die Studiengänge Informatik und Elektrotechnik am Fachbreich 03 Elektrotechnik und Informatik aufgenommen. So werden dort im fünften Semester in der Veranstaltung IT-Security grundlegenden Kompetenzen zum Thema IT-Sicherheit vermittelt, welche einem allgemeinen Anspruch genügen.\cite{hochschuleniederrheinModulhandbuchVollzeitBA2019}