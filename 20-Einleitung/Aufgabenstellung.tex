\section{Aufgabenstellung}
\label{sec:Aufgabenstellung}
Das Praktikum wird durch ein Auswertungs- und Überwachungssystem begleitet, das eine objektiv nachvollziehbare Bewertung vornehmen kann und die in den Bewertungsprozess eingeflossenen Parameter dokumentiert. \cite[S. 2]{sosnaKonzeptionUndRealisierung2010}

Ziel meiner Arbeit ist die Modernisierung und Verbesserung dieses Auswertungs- und Überwachungssystems. Anforderung hierbei war, dass das neue System die gleichen Basisfunktionalitäten wie das Altsystem erfüllen muss.

In der einführenden Betrachtung (\autoref{chap:Analyse}) wird der aktuelle Stand des Systems, Schnittstellen zwischen Server und Client sowie der Begründung für die Veränderung dargelegt. Aus dieser einführenden Betrachtung werden dann Anforderungen abgeleitet.

Im folgenden \autoref{chap:Entwurf} werden Entwürfe für die verschiedenen, neu designten Komponenten des Servers erstellt. 

Anhand der abgeleiteten Anforderungen und des Entwurfs der verschiedenen Komponenten werden im \autoref{chap:Technologien} diverse Technologien diskutiert und passende ausgewählt.

Die Implementierung des Entwurfs mit den gewählten Technologien wird im \autoref{chap:Realisierung} beschrieben.

Eine kritische Auseinandersetzung mit dem Ergebnis dieser Arbeit folgt und es werden Aussichten für mögliche Veränderungen und Erweiterungen gegeben.

Aus diesem Aufbau der Arbeit ergeben sich die folgenden Aufgaben:
\begin{itemize}
	\item Analyse des vorliegenden Systems
	\item Analyse der Voraussetzungen
	\item Ableitung von Anforderungen
	\item Entwurf der Architektur
	\item Entwurf der einzelnen Komponenten
	\item Diskussion einzusetzender Technologien
	\item Implementierung des Entwurfs
	\item Bereitstellung von Installations- und Bedienungsanleitung.
\end{itemize}