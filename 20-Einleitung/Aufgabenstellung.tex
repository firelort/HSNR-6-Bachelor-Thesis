\section{Aufgabenstellung}
\label{sec:Aufgabenstellung}
Begleitend zu der Veranstaltung IT-Security für die Studiengänge Bachelor Informatik und den Bachelor Elektrotechnik des Fachbereichs 03 Elektrotechnik und Informatik der Hochschule Niederrhein werden 3 Praktika durchgeführt. Diese sollen den Studierenden praktisch Erfahrungen ermöglichen.

Das zweite Praktikum \textquote{Catch me, if you can} stellt einen Vergleichswettbewerb dar. An diesem Wettbewerb nehmen mehrere Teams teil, welche sich alle in einem gemeinsamen Netzwerk befinden. Die Aufgabe der Teams besteht darin, festgelegte IT-Dienste (abgesichert) bereit zustellen, geheime Informationen sowohl auf dem eigenen Rechner als auch auf den Rechnern der anderen Teams zu finden und zu verhindern, dass andere Teams an die eigenen geheimen Informationen gelangen.\cite[S. 2]{sosnaKonzeptionUndRealisierung2010} Die geheimen Informationen sind logisch gesehen Passwörter oder private Bilder und werden durch sogenannte Flags repräsentiert. Eine Flag ist ein gehashter Zeichenfolge und hat immer die gleiche Länge.

Das Praktikum wird durch ein Auswertungs- und Überwachungssystem überwacht - anderes Wort -, welches eine objektiv nachvollziehbare Bewertung vornehmen kann und die in den Bewertungsprozess eingeflossenen Parameter dokumentiert.\cite[S. 2]{sosnaKonzeptionUndRealisierung2010}

Ziel meiner Arbeit ist die Modernisierung und Verbesserung dieses Auswertungs- und Überwachungssystems.

In der einführenden Betrachtung (\autoref{chap:Analyse}) wird der aktuelle Stand des Systems, Schnittstellen zwischen Server und Client sowie der Begründung für die Veränderung dargelegt. 

Aus dieser einführenden Betrachtung werden dann im \autoref{chap:Entwurf} Entwurf Anforderungen abgeleitet und Entwürfe für die verschieden Komponenten des Servers erstellt. 

An Hand der abgeleiteten Anforderungen und des Entwurfs der verschiedenen Komponenten wird im \autoref{chap:Technologien} Technologien verschiedene Technologien diskutiert und passende Technologien ausgewählt.

Die Implementierung des Entwurfs mit den gewählten Technologien wird im \autoref{chap:Realisierung} Realisierung beschrieben.

Eine kritische Auseinandersetzung mit dem Ergebnis dieser Arbeit folgt und es werden Aussichten für mögliche Veränderungen und Verbesserungen gegeben.