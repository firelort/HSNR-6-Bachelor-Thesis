\section{Aufgabenstellung}
\label{sec:Aufgabenstellung}
Begleitend zu der Veranstaltung IT-Sicherheit für die Studiengänge Bachelor Informatik und den Bachelor Elektrotechnik des Fachbereichs 03 Elektrotechnik und Informatik der Hochschule Niederrhein werden 3 Versuche im Rahmen des Praktikums durchgeführt. Diese sollen den Studierenden praktische Erfahrungen ermöglichen.

Der zweite Versuch namens \textquote{Catch me, if you can} stellt einen Vergleichswettbewerb dar. An diesem Wettbewerb nehmen mehrere Teams teil, welche sich alle in einem gemeinsamen Computernetzwerk befinden. Die Aufgabe der Teams besteht darin, festgelegte Programme/Dienste abgesichert bereit zustellen, geheime Informationen sowohl auf dem eigenen Rechner als auch auf den Rechnern der anderen Teams zu finden und Schwachstellen abzusichern, um so zu verhindern, dass andere Teams an die eigenen geheimen Informationen gelangen.\cite[S. 2]{sosnaKonzeptionUndRealisierung2010} Die geheimen Informationen sind logisch gesehen Passwörter oder private Bilder und werden durch sogenannte Flags repräsentiert. Eine Flag ist eine generierte Zeichenfolge mit fester Länge.

Das Praktikum wird durch ein Auswertungs- und Überwachungssystem begleitet, welches eine objektiv nachvollziehbare Bewertung vornehmen kann und die in den Bewertungsprozess eingeflossenen Parameter dokumentiert.\cite[S. 2]{sosnaKonzeptionUndRealisierung2010}

Ziel meiner Arbeit ist die Modernisierung und Verbesserung dieses Auswertungs- und Überwachungssystems.

In der einführenden Betrachtung (\autoref{chap:Analyse}) wird der aktuelle Stand des Systems, Schnittstellen zwischen Server und Client sowie der Begründung für die Veränderung dargelegt. Aus dieser einführenden Betrachtung werden dann Anforderungen abgeleitet.

Im Folgenden \autoref{chap:Entwurf} werden Entwürfe für die verschieden Komponenten des Servers erstellt. 

Anhand der abgeleiteten Anforderungen und des Entwurfs der verschiedenen Komponenten werden im \autoref{chap:Technologien} verschiedene Technologien diskutiert und passende ausgewählt.

Die Implementierung des Entwurfs mit den gewählten Technologien wird im \autoref{chap:Realisierung} beschrieben.

Eine kritische Auseinandersetzung mit dem Ergebnis dieser Arbeit folgt und es werden Aussichten für mögliche Veränderungen und Erweiterungen gegeben.