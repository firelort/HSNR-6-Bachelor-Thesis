\section{Backend} \label{sec:Backend}
Bei der Entwicklung des Webservers wird auf ein Backend-Webframework zurückgegriffen. Dieses erleichtert die Entwicklung und Wartung der Webanwendung in dem Werkzeuge und Bibliotheken zur Verfügung gestellt werden, welche allgemeine Aufgabe übernehmen. Zu diesen allgemeinen Aufgaben gehört das Verarbeiten von HTTP Anfragen, das Erstellen von HTTP Antworten und das Weiterleiten von Anfragen an die entsprechende Funktion. Bei der Implementierung der Logik müssen alle diese allgemeinen Aufgaben nicht mehr beachtet werden.

Um die Weiterentwicklung durch Studierende des Bachelor Informatik der Hochschule Niederrhein zu gewährleisten, werden nur Webframeworks, welche eine Programmierung in C, C++, Python oder JavaScript vorsehen, in Betracht gezogen. Da eben diese an der Hochschule Niederrhein gelehrt werden. Derzeitig sind neben Frameworks wie Spring (Java), Ruby on Rails (Ruby) und Laravel (PHP), welche nicht in den Vergleich aufgenommen werden, Django (Python), Flask(Python) und Express(Node.js/JavaScript) verbreitet.\cite{mdncontributorsServersideWebFrameworks2020}

In der Lehrveranstaltung \textit{Web-Engineering} wird die Entwicklung mehrere Backend-Server unter Zuhilfenahme von Python durchgeführt. Deshalb werden im folgenden nur Django und Flask betrachtet. 

Anschließend wird ein Framework gewählt, mit dem der Backend-Server realisiert werden soll.

\subsection{Django}\label{subsec:Django}
Django folgt der sogenannten \textit{Batteries Included} Philosophie. 

Bei dieser werden vielseitige Standardbibliotheken mit ausgeliefert, so dass ein Nutzer keine oder wenige separate Pakete herunterladen muss.\cite{kuchlingPEP206Python}

Django liefert unter anderem Modulen für Caching, Logging, Versenden von E-Mails, RSS-Feeds, Pagination, Security und der automatischen Generierung von Administrationsseiten. \cite{djangoDjangoDocumentationDjango} Durch den \textit{Batteries Included} Ansatzes funktionieren die Komponenten reibungslos untereinander und sind mit dem Kern kompatibel. Des Weiteren sind die Dokumentationen der verschiedenen Module zentrale an einem Ort verfügbar. 

Django wurde von einem Team entwickelt, welches ursprünglich Zeitungs-Websites erstellt und verwaltet hat. Dieses Team hat die gemeinsame Codebasis abstrahiert und in ein Framework überführt. Daher ist Django besonders, aber nicht nur, für Nachrichtenseiten und Content Managment Systeme (CMS) geeignet. \cite{mdncontributorsDjangoIntroduction2019}

\subsection{Flask}
Im Gegensatz zu Django ist Flask ein Mikroframework und verfolgt die \textit{Batteries Included} Philosophie nicht. Bei einem Mikroframework ist der Kern einfach, aber erweiterbar gestaltet und es gibt wenige bis keine Abhängigkeiten zu externen Bibliotheken. Dieser Ansatz überträgt der programmierenden Person mehr Verantwortung aber auch mehr Freiheiten. So trifft Flask beispielsweise nicht die Entscheidung, welche Datenbank genutzt werden soll. \cite{palletsForewordFlaskDocumentation2010} 

Wie bei Django (\ref{subsec:Django}) werden zwei Module bei der Installation von Flask mit geladen. Zum einen wird auf das Modul \textit{Werkzeug} zurückgegriffen, um eine ordnungsgemäße WSGI-Anwendung zu ermöglichen. Zum anderen wird Jinja2 mitgeliefert, da die meisten Anwendungen eine Template-Engine benötigen.
Flask will ein solide Grundlage für alle Anwendung sein, bei der die entwickelnde Person viele Freiheiten genießt.\cite{palletsDesignDecisionsFlask2010}

\subsection{Wahl des Frameworks}

Das Mikroframework Flask wird zur Implementierung des Webservers genutzt. Die Anwendung wird in Python geschrieben und kann daher durch Studierende weiterentwickelt werden, ohne dass diese sich in eine neue Programmiersprache einarbeiten müssen. Des Weiteren werden viele Module von Django nicht benötigt. Diese stellen einen Overhead dar. Auch handelt es sich bei der zu entwickelnden Software um eine vergleichbar kleine Anwendung. Es würden nur wenige Vorteile von Django genutzt. Flask kann mit Hilfe von Community Modulen bei Bedarf erweitert werden.